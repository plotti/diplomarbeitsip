% ---------------------------------------
%
%    Beispieldiplomarbeit
%
%    - text kann gel{\"o}scht, und mit eigenen 
%    Inhalten gef{\"u}llt werden.
%
% ---------------------------------------

\documentclass[
    smallheadings,  % kleinere {\"U}berschriften
    oneside,        % einseitig, nur rechte seiten
    liststotoc,     % listen in inhaltsverzeichnis aufnehmen
    bibtotoc,       % literaturverzeichnis in inhltsvz. aufnehmen
    headsepline,     % trennlinie unter kopfzeile
    12pt
    ]{scrbook}

\usepackage{a4}  %a4 Seitenformat benutzen
\usepackage[ngerman,english]{babel} %Verwende deutsche, bzw. amerikanische Silbentrennung
%\usepackage[utf8]{inputenc} %damit k{\"o}nnen Umlaute ganz normal geschrieben werden. 
\usepackage[ansinew]{inputenc}

\usepackage{subfigure} %f{\"u}r mehrteilige Grafiken
\usepackage{epsfig}    %damit funktioniert das einbinden von grafiken {\"u}ber epsfig.
\usepackage{graphicx}     % zum einbinden von grafiken
\graphicspath{{grafiken}{../}{kapitel}} %da sind m{\"o}gliche bilder fuer den includegraphics-Befehl zu finden (man muss dann nicht den ganzen Pfad bei includegraphics angeben. 

\usepackage{multirow}     %fuer kompliziertere Tabellen
\usepackage{framed}
\usepackage{scrpage2}     % paket f{\"u}r kopf- und fu{\ss}zeilen
\pagestyle{scrheadings}   % kopzeilenseitenstil
\usepackage{natbib}       % Literaturverzeichnis
\usepackage{setspace}
\usepackage{url}          % fuer urls: schreibweise ist z.B. \url{http://www.uni-mannheim.de}


\setlength{\parindent}{0pt}
\setlength{\parskip}\medskipamount % besser als explizite Angabe in pt

\onehalfspacing


% kapitel{\"u}berschriften in schriftart mit serifen
\setkomafont{sectioning}{\normalfont\normalcolor\bfseries}

% gestaltung der kopfzeilen
\ohead{\pagemark}
\cfoot{}
\cohead{}
\ihead{\headmark}
\setkomafont{pagehead}{\normalfont\bfseries}
\setkomafont{pagenumber}{\normalfont\bfseries}
\automark{section}

% ----- ende der pr{\"a}ambel ----------------------------------

\begin{document}  % dokument f{\"a}ngt an
\selectlanguage{ngerman} %deutsche Silbentrennung
\frontmatter      % vorspann, kapitel r{\"o}misch nummeriert

% Die Titelseite der Arbeit

\begin{titlepage}

\begin{center} % zentrieren

  % Logo der Universit{\"a}t Mannheim
  \begin{figure}[ht]
    \centering
    \includegraphics{grafiken/unilogo}
  \end{figure}
  
  % Vertikaler Zwischenraum
  \bigskip
  \vfill 
  \begin{framed}
  % Titel der Arbeit und Typ der Arbeit, umrandet
    \begin{center}
      \textsc{{\Large Hier steht der Titel der Arbeit \\ M{\"o}glicher Untertitel\\}}
                                % Letztes \\ ist wichtig, beginnt eine neue Zeile f{\"u}r die Art der Arbeit
  
      \bigskip
  
                                % Art der Arbeit, ggf. auszutauschen gegen Seminar- oder Doktorarbeit
      \textbf{Diplomarbeit}
    \end{center}
    \end{framed}
    \vfill
    \vfill
  
  % Daten des Erstellers, Einreichungsdatum
  % in einer Tabelle ausgerichtet
  \begin{tabular*}{0.62\textwidth}{r@{\extracolsep{\fill}}l}
    eingereicht im: & Januar 2004\\\\
    von: & Martin Mustermann\\
    & geboren am 23.  M{\"a}rz 1979\\
    & in Mannheim\\
    \\
    Matrikelnummer: & 012345\\
  \end{tabular*}
  \vfill
  \vfill
  
  % Unten: Kontaktdaten des Lehrstuhls f{\"u}r Wirtschaftsinformatik 1
  
  \rule{\textwidth}{.4pt}\\ % vertikale Linie
  Universit{\"a}t Mannheim\\
  Lehrstuhl f{\"u}r ABWL und Wirtschaftsinformatik\\
  D -- 68131 Mannheim\\
  Telefon: +49 621 1811691, Fax +49 621 1811692\\
  Internet: \url{http://www.bwl.uni-mannheim.de/wifo1}
\end{center}

\end{titlepage} % Ende des Titelblatts

%%% Local Variables: 
%%% mode: latex
%%% TeX-master: "~/Documents/DA-Vorlage/beispiel/da-beispiel"
%%% End: 
     % titelseite einbinden
\tableofcontents            % inhaltsverzeichnis
%-------------------------------------
%
%    Stichwortverzeichnis (ca. 5-10 Stichworte, welche den Inhalt der Arbeit beschreiben
%
%-------------------------------------

\chapter{Stichwortverzeichnis} % beachte addchap
\begin{labeling}{1234567890}
        \item Voice over IP
        \item Sprachkommunikation in Computerspielen
        \item Session Initiation Protocol
        \item Proxemik Zonen
        \item Distanzbasierte Sprachverbindungen
        \item Peer-to-peer Spiele
\end{labeling} 


\begin{sloppypar}
	
\end{sloppypar}  % Stichwortverzeichnis
\listoffigures              % abbildungsverzeichnis
\listoftables               % tabellenverzeichnis
%-------------------------------------
%
%    minimales abk�rzungsverzeichnis
%
%-------------------------------------

\addchap{Abk\"{u}rzungsverzeichnis} % beachte addchap
\begin{labeling}{1234567890}
\item[3GPP] 3rd Generation Partnership Project
\item[API] Application Programming Interface
\item[CAN] Content Addressable Network 
\item[CMC] Computer-mediated communication,
\item[CSCP] Computer Supported Cooperative Play
\item[CSRC] Computer Security Resource Center
\item[DHT] Distributed hash table
\item[DICE] Distributed Immersive Communication Environment
\item[DPM] Distributed Partial Mixing
\item[DSL] Digital Subscriber Line
\item[FGW] Forschungsgruppe Wahlen
\item[FPS] First-person shooter
\item[HRTF] Head-related Transfer Function
\item[HTTP] Hypertext Transfer Protocol
\item[ICE] Interactive Connectivity Establishment
\item[I-ETS] Interim European Telecommunications Standard 
\item[IMPS] Instant Messaging and Presence Services
\item[IMTC] International Multimedia Telecommunications Consortium
\item[IP] Internet-Protokoll
\item[IPA] Integrated Payload Analysis
\item[ITU-T] International Telecommunication Union-Telecommunication
\item[IVE] Immersive Virtual Environments
\item[LAN] Local Area Network
\item[Megaco] Media Gateway Control (IETF working group)
\item[MGCP] Media Gateway Control Protocol
\item[MICE] Mobile Immersive Communication Environment 
\item[MMORPG] Massively Multiplayer Online Role-Playing Game
\item[MOS] Mean Opinion Score
\item[MUD] Multi-User Dungeon 
\item[NAT Network Address Translation
\item[NTP] Network Time Protocol
\item[P2P] Peer-to-Peer
\item[PCM] Pulse-Code Modulation 
\item[PINT] PSTN Internet Internetworking
\item[PT] Payload Type
\item[PwC] PricewaterhouseCoopers 
\item[RFC] Request For Comment
\item[RR] Receiver Report
\item[RR] Sender Report
\item[RTP] Real-Time Protocol
\item[SDES] Source Description
\item[SDP] Session Data Protocol
\item[SIP] Session Initiation Protocol
\item[SIMPLE] Session Initiation Protocol for Instant Presence Leveraging Extensions
\item[SSRC] Synchronization Source
\item[STUN] Simple Traversal of UDP Through NATs
\item[TCP] Transmission Control Protocol
\item[PTSN] Public Telephone Switched Network
\item[TURN] Traversal Using Relay NAT
\item[UA] User Agent
\item[UAC] User Agent Client
\item[UAS] User Agent Server
\item[UDP] User Datagram Protocol
\item[URI] Uniform Resource Identifier
\item[URL] Uniform Resource Locator 
\item[VoIP] Voice over Internet Protocol
\item[WoW] World of Warcraft 
\item[XMPP] Extensible Messaging and Presence Protocol
\end{labeling}   % beispiel eines handerstellten
                                %verzeichnisses
\mainmatter       % hauptteil, kapitel lateinisch nummeriert
\chapter{Einf�hrung}
\label{chap:einleitung}
%Dieses Kapitel gibt zun�chst einen Einblick auf Entwicklung der Sprachkommunikation in Mehrspieler-Computerspielen, die neue Herausforderungen das Kommunikationsmedium stellen. Anhand dieser wird die Motivation f�r die Arbeit hergeleitet und die Marktrelevanz dieses Themas verdeutlicht. Abschlie�end werden konkrete Ziele f�r eine Umsetzung einer distanzbasierten Sprachkommunikation gestellt und eine �bersicht �ber den Aufbau der Arbeit gegeben.

%\section{Virtuelle Welten}

\section{Motivation}
Aktuelle 3D-Mehrspieler-Computerspiele betonen vor allem die Interaktion des Spielers mit dem Spiel, indem die 3D-Welten immer fotorealistischer und detailgetreuer werden. Die Sprachkommunikation der Spieler untereinander ist allerdings zum gro�en Teil nicht in diese Spiele integriert und wird nur mit Hilfe von unflexiblen, st�ranf�lligen und kostenpflichtigen Drittanbieter-L�sungen erm�glicht, bei denen die Konversation nicht mit den Avataren auf dem Bildschirm des Spielers verkn�pft ist. 

Durch die rasante Entwicklung von Mehrspieler-Computerspielen sind mittlerweile Millionen\footnote{Blizzard Entertainment: World of Warcraft User Statistics, Seite besucht 15.03.2008, http://blizzard.co.uk/press/080122.shtml} von geografisch verteilten Nutzern in der Lage miteinander zu spielen und zu kommunizieren. Neben dem klassischen konkurrenzbetonten Charakter tritt auch der soziale Charakter von Spielen, die in virtuellen Welten wie There\footnote{Makena Technologies, Seite besucht 22.03.2008, http://www.there.com/}, Active Worlds\footnote{Activeworlds Inc., Seite besucht 22.03.2008, http://www.activeworlds.com/} und Second Life\footnote{LindenResearch Inc. , Seite besucht 22.03.2008, http://secondlife.com/} stattfinden, immer mehr in den Vordergrund. 

Lange Zeit war die Textkommunikation das einzige integrierte Kommunikationsmedium, das eine Verst�ndigung in Mehrspieler-Computerspielen erlaubte. Textkommunikation stellte aber aufgrund ihrer nicht ausreichenden Medienreichhaltigkeit (Mediarichness\footnote{Siehe Abschnitt \ref{mediarichness}}) \citep{rice92}, ein unzureichendes Kommunikationsmedium dar, um komplizierte Telekooperationsaufgaben, wie z. B. das taktische Befehligen von mehreren Spielern in Echtzeit zu l�sen \cite{carter03}, \cite{pena04}, \cite{thon06}. Deswegen wurde sie zunehmend von der Sprachkommunikation abgel�st. 

Auf der Seite der Spielehersteller wird jedoch eine integrierte Sprachkommunikation bei den wenigsten Spielen unterst�tzt, da der Betrieb serverbasierter Konferenzserver f�r alle Spieler mit immensen Betreiberkosten f�r das Datenaufkommen und die Rechenleistung verbunden ist. Aufgrund des hohen Bedarfs an derartigen L�sungen bieten Drittanbieter seit einigen Jahren erg�nzende Konferenzl�sungen speziell f�r Spiele an. Diese sehen vor, dass dedizierte Server mit einer hohen Bandbreite und Rechenleistung von Spielergruppen eingekauft werden, um darauf Audiokonferenzen abzuhalten. Da Sprache eine schnelle, pr�zise und einfache Koordination bei komplexen Aufgabenstellungen untereinander erlaubt, wurde ihr Einsatz zunehmend f�r den Spielerfolg entscheidend. Diese Tatsache f�hrte letztendlich zu einer schnellen Verbreitung und Popularit�t solcher L�sungen. 

Der Hauptnachteil solcher L�sungen liegt vor allem darin, dass sie nicht in das Spiel integriert sind und daher kein unmittelbarer Zusammenhang zwischen den Teilnehmern einer Konferenz und den Teilnehmern des Spieles besteht. Die Teilnehmer solcher Massenkonferenzen sind nicht in der Lage, zu bestimmen, welchen Spieler sie adressieren m�chten und haben auch keine Kontrolle dar�ber, wessen Audiosignal sie empfangen m�chten. Benutzer finden es schwer, das Gesprochene in den Kopfh�rern mit den Avataren auf ihren Monitoren zu verkn�pfen. Da alle Spieler einen gemeinsamen Sprachkanal benutzen, wird dieser schnell unverst�ndlich, wenn zu viele Teilnehmer gleichzeitig sprechen. Die Teilnahme an solchen Konferenzen muss schon vor dem Spiel konfiguriert werden und erlaubt deswegen auch keine spontane Kommunikation mit einem beliebigen Teilnehmer w�hrend des Spiels. Da die meisten Anwender dar�ber hinaus selbst kaum �ber die Bandbreiten- und Rechenkapazit�ten verf�gen, um privat solche Konferenzserver zu betreiben, sind sie auf den Einsatz kostenpflichtiger dedizierter Konferenzserver angewiesen. 

Der Eifer, mit der die Sprachkommunikation f�r Mehrspieler-Computerspiele trotz dieser Probleme adoptiert wurde, zeigt eindeutig wie hoch der Bedarf an solchen L�sungen ist. Bei den potenziellen Nutzern derartiger Anwendungen spricht man von einer Spielergemeinde, die der Industrie allein in Deutschland einen Umsatz von 2,14 Milliarden Euro beschert hat. Es wird erwartet, dass der Umsatz noch im laufenden Jahr jenen der Musikbranche �bertreffen wird \cite{ErnstYoung:07}, \cite{pwc:07}. 

Obwohl die Steuerung von Konferenzen mittlerweile seit Jahren erforscht wird, existieren bislang keine innovativen Methoden, die bei einem so gro�en Markt Anwendung gefunden h�tten. Dies ist darauf zur�ckzuf�hren, dass die oben genannten Probleme zwar weitgehend durch medienpsychologische Studien bekannt sind, dennoch nahezu keine Schnittmenge aus existierenden Ans�tzen vorhanden ist, die sich mit der technischen Seite der Realisierung von Audiokonferenzen und den Bed�rfnissen der Spielergemeinde befasst. Die Industrie hingegen verl�sst sich auf propriet�re klassische L�sungen, die nicht in der Lage sind, die genannten Probleme zu adressieren.

\section{Ziel der Diplomarbeit}

In dieser Diplomarbeit wird eine Implementierung einer integrierten Sprachkommunikation f�r Spiele entwickelt, die im Gegensatz zu bisherigen L�sungen keinen zentralen Konferenz-Server zum Mischen des Audiostroms benutzt. Die Audioverbindung erfolgt auf Peer-to-Peer-Basis, wobei nach M�glichkeit bereits bestehende VoIP-Komponenten verwendet werden. Ein solcher Ansatz soll nicht Kosteneinsparungen erm�glichen, indem die Bandbreite und Rechenleistung von Teilnehmern gestellt wird, sondern ihnen auch eine bessere Kontrolle �ber Audiokonferenzen geben. 

Anhand eines Prototyps wird untersucht, wie eine integrierte Sprach\-kommuni\-kations\-l�sung sinnvoll in Computerspielen eingesetzt werden kann. Dazu werden im theoretischen Teil der Arbeit bisherige Studien der Sprachkommunikation ausgewertet und sukzessive Verbesserungsvorschl�ge abgeleitet. Dar�ber hinaus wird beim Entwurf einer solchen L�sung der Einfluss verschiedener Netzwerkarchitekturen und Sprach\-�bertragungs\-standards auf die Um\-setz\-bar\-keit ber�ck\-sichtigt. Als Echtzeit-Spiel wird eine offene 3D-Engine genutzt, die es dem Spieler erlaubt, sich in einer 3D-Welt mit seinen Mitspielern zu bewegen und mit ihnen in Kontakt zu treten.  

Im Gegensatz zu bisherigen Alternativen, die eine feste Auswahl von Konferenz\-teilnehmern vorsehen, ist das Ziel, die Sprach\-kommunikation derart zu implementieren, dass der Anwender spontan im Spiel mit jedem seiner Mitspieler kommunizieren kann. Dazu werden zwischen Mitspielern dynamisch Konferenzen aufgebaut, falls sich diese in einer zu definierenden "H�rn�he" befinden. Sobald sie diese wieder verlassen, findet ein Abbau entsprechender Konferenzen statt. Hierbei gilt es zu �berlegen, verschiedene Verbindungsstufen aufrechtzuerhalten. Im Schlussteil der Arbeit werden die Auswirkungen einer solchen L�sung untersucht und Konsequenzen abgeleitet. 

%Ein solches Szenario erm�glicht sowohl eine interpersonelle Kommunikation zwischen Mitspielern, indem sich nur 2 Spieler in H�rn�he befinden, als auch eine Massenkommunikation, bei der mehrere Spieler in gemeinsamer N�he miteinander sprechen k�nnen.

\section{Aufbau der Arbeit}
\label{sec:aufbau-der-arbeit}

Die vorliegende Arbeit gliedert sich wie folgt:

Zun�chst werden in \textit{Kapitel 2} die Grundlagen der Kommunikation vorgestellt. Diese werden in \textit{Kapitel 3} genutzt, um die Ergebnisse von Untersuchungen bisheriger Sprachkommunikationsl�sungen einzugliedern, gemeinsame Ursachen f�r Probleme zu finden und innovative Konzepte f�r eine bessere L�sung zu erarbeiten. So sollen in dieser Arbeit nicht nur technische Herausforderungen einer integrierten Sprach\-kommunikations\-l�sung diskutiert werden, sondern auch Schw�chen in der Benutzer\-freundlichkeit bisheriger Programme durch die Einf�hrung von Sprach\-�bertragungs\-metaphern und Mehrkanal-Modellen kompensiert werden. 

Die f�r eine technische Umsetzung notwendigen Grundlagen und wichtige Bestandteile des gew�hlten Protokolls werden in Kapitel \textit{Kapitel 4} vorgestellt. \textit{Kapitel 5} bietet eine �bersicht relevanter Arbeiten im Bereich der Sprachkommunikation in Mehrspieler-Computerspielen und zeigt ihre Vor- und Nachteile. Dieses Kapitel erfordert technische VoIP-Grundlagen aus \textit{Kapitel 4}. 

Da die Umsetzung einer Sprachkommunikation auch an die darunter liegende Architektur gekn�pft ist, aber noch keine strukturierte Analyse von SIP-Protokoll-basierten L�sungen und ihrer Tauglichkeit f�r Sprachkommunikationsl�sungen existiert, wird diese in \textit{Kapitel 6} vorgenommen und eine Empfehlung f�r die Implementierung eines hybriden Unicast-Ansatzes getroffen. Dieses Kapitel geht auch auf die resultierenden Bandbreitenprobleme der gew�hlten Architektur ein und stellt eigene Konzepte vor, wie anhand eines Partial-Mesh-Ansatzes, bei dem nur relevante Audioverbindungen etabliert werden, Bandbreite eingespart werden kann. 

\textit{Kapitel 7} befasst sich mit der Implementierung eines Spiel-Prototyps auf Basis eines hybriden SIP Unicast-Ansatzes und pr�sentiert ausgew�hlte Details und Probleme bei der Umsetzung. Hauptaugenmerk liegt dabei auf der technischen Umsetzung der in \textit{Kapitel 3} vorgestellten distanzbasierten Sprachkommunikation, dem Einfluss der Zonenkonzepte auf die verwendete Bandbreite und der Zusammenarbeit der 3D-Welt mit der VoIP-Anwendung. 

Da sich bereits in \textit{Kapitel 6 und 7} abzeichnet, dass das gew�hlte Protokoll auch f�r mehr als VoIP nutzbar ist, wird in \textit{Kapitel 8} gezeigt, wie das gew�hlte SIP-Protokoll als Netzwerkschicht eingesetzt werden kann. Dabei wird auch die Implementierung der SIP-basierten Netzwerkschnittstelle des Prototypen erl�utert.

In Kapitel \textit{Kapitel 9} wird das Resultat der Umsetzung anhand von Messergebnissen erfasst. \textit{Kapitel 10} fasst die Aufgaben, die in dieser Arbeit geleistet wurden, noch einmal zusammen und geht auf noch offene Fragen und Verbesserungs\-m�glichkeiten ein.

%\section{Zwei schnell wachsende M�rkte: VoIP und Computerspiele}
%\section{Von "`Tennis for Two"' bis zur Kommunikationsplattform}
%Ihre hohe Beliebtheit und Marktrelevanz haben Computerspiele vor allem ihrem stetigen Wandel zu verdanken. Sie entwickelten sich vom akademischen, nichtkommerziellen Nebenprodukt zu einer ausgereiften kommerziellen Spiele- und Kommunikationsplatform. Die akademischen Anf�nge finden sich im Jahr 1958, als der Physiker Willy Higinbotham das Spiel "`Tennis for Two"' entwickelte, welches auf einem Osziloskop gespielt wurde. Diesem sollte im Jahre 1961 das Spiel "`Spacewar"' folgen, das an der Universit�t Stanford von Steve Russell entwickelt wurde \cite{CHM}. Erst weitere Jahre danach entstanden die ersten kommerziellen Computerspiele. "`Pong"' das im Jahre 1972 auf dem Atari System ver�ffentlicht wurde, wurde zur Sensation und zum ersten kommerziellen Erfolg. Als kommerzielle Produkte wurden Computerspiele in den 80er Jahren haupts�chlich in Spielhallen gespielt oder von Spielern auf einfachen Konsolen zu Hause. Dabei besa�en die bekanntesten Vertreter der ersten Generation wie Pacman oder Space Invaders nur einen Einzelspieler Modus. Das einzige Ziel des Spiels war das Erreichen der H�chstpunktzahl. 
%Ende der 80er Jahre verlagerte sich das Zentrum der Spieleindustrie nach Japan wo die ersten Generationen von Konsolenspielen gro�en kommerziellen Erfolg erreichten. Ein Hauptfaktor f�r den Erfolg war vor allem die M�glichkeit zuhause mit mehreren Freunden zeitgleich gegeneinadner zu spielen. Mitte der 90 er Jahre wurde der Personal Computer als Spieleplatform entdeckt. Vor allem durch das Konzept der Tastatur und Maus wurden neue Spielegenres erm�glicht, die durch durch eine "`point and click "' Steuerung mit Hilfe der Maus durch ihre Bedienfreundlichkeit �berzeugten.  Zwar hatten sich Tastatur und Maus sich als sehr ergonomisches "`Single-Player-Interface"' behauptet, erm�glichten jedoch im Vergleich zu Spielekonsolen nur sehr eingeschr�nkte lokale Mehrspieler M�glichkeiten. 

%triggerhappy buch als quelle zu schwach.
%TODO ABSTRACT. 

%\section{Alte Motivation}
%Mit dem Aufkommen von Mehrspieler-Spielen wurde es auch zum ersten Mal n�tig die Spielezust�nde der Mitspieler miteinander auszutauschen. Dies f�hrte zur Einf�hrung der Client- und Server-Architektur, in der ein zentraler Server die Spielelogik verwaltete und an Clients verteilte. Da man sich in LANs befand, musste ein solcher Server die Daten von 2 bis zu 1000 Spielern verwalten. 
%Mundane metaverses communicating vy voice in virtual worlds ---> INTRO
%\cite{wadley08}

%\section{Altes Ziel der Arbeit}
%In dieser Diplomarbeit sollen ein P2P basiertes Verfahren zur Sprachkommunikation f�r Spiele entwickelt werden. Statt propri�teren L�sungen wie Skype, soll ein offenes und bereits etabliertes Protokoll genutzt werden. Dabei sollen und Probleme und Chancen des Einsatzes eines solchen Protokolls zu er�rtert und L�sungsvorschl�ge erarbeitet werden. Zum Einsatz soll das SIP Protokoll kommen, das bisher im Bereich des VoIP zu einem offenen Standard entwickelt hat, der auf Software- und Hardware-Ebene seit mehreren Jahren genutzt wird. Obwohl es sich im Ansatz um ein Client-Server-rotokoll handelt, bietet SIP zahlreiche M�glichkeiten es auch im P2P Modus zu betreiben, und es finden sich viele An�tze das Protokoll komplett dezentral einzusetzen.  
%Bisher ist der Einsatz von SIP im Bereich von Multispieler-Spielen noch nicht tiefgehend erforscht. Es soll  untersucht werden wie das SIP-Protokoll sinnvoll f�r eine standardisierte P2P-Schnittstelle zwischen Clients eingesetzt werden kann. Ebenfalls soll der zuk�nftige Einsatz von P2PSIP soll untersucht werden...TODO.
%Es soll eine Spiele-Implementierung entwickelt werden, die mit dem SIP Protokoll, den Datenaustausch zwischen Spielern erm�glicht. Dieser soll mit Hilfe des SIP-SIMPLE-Protokolls implementiert werden.Die Mehrspieler-Kommunikation, die bisher mit propri�terer Software gel�st wird und nicht mit der Spielelogik verzahnt ist, soll im Rahmen der Diplomarbeit in das Spiel integriert werden. So sollen zwischen Mitspielern dynamisch Konferenzen aufgebaut werden, falls sich diese in einer zu definierten "H�rn�he" befinden. Falls sich ein Mitspieler au�erhalb der H�rn�he befindet, sollen entsprechende Konferenzen abgebaut werden, wobei zu �berlegen ist, aber verschiedene Verbindungsstufen aufrechtzuerhalten. Zur Abblidung von Lokalen Audioereignissen soll eine weiteren Audiorendering-Engine in das Spiel integriert werden, die es erm�glichen kann  lokale Soundquellen zu positionieren und Audiodaten, die nicht �ber das Netzwerk zu �bertragen sind, lokal zu verwalten. Die Implementierung soll mit Skype und Multicast-Sockets Netzwerklayern verglichen werden, es soll untersucht werden inwiefern die verschiedenen zugrunde liegenden Systeme miteinander vergleichbar sind. Unter diesem Aspekt soll analysiert werden, wie sich die Performace dieser Systeme in einem direkten Benchmark miteinander verh�lt.

%\subsection{Komplett neue Intro}
%Virtuelle Welten
%Voip in Virtuellen Welten
%DAS PROBLEM:
%DIE L�SUNG:

%%% Local Variables: 
%%% mode: latex
%%% TeX-master: "..\\da-beispiel"
%%% End: 
\cleardoublepage

\chapter{Computerspiele}

\section{Zwei schnell wachsende M�rkte: VoIP und Computerspiele}

Eine von Ipoque \cite{ipoque:2007} durchgef�hrte Studie bei der drei Petabyte anonymer Daten, erhoben von mehr als einer Million Nutzern in Australien, Deutschland, dem Nahen Osten, Ost- und S�deuropa, in die Auswertung eingeflossen sind zeigt, dass P2P Anwendungen im Internet mehr Verkehr als alle anderen Anwendungen zusammen erzeugen. Dabei zeigt die Studie, dass Voice over IP (VoIP) mittlerweile zu einer weitgenutzten Anwendung geworden, nicht zuletzt aufgrund des enormen Erfolges von Skype mit seiner einfachen Nutzung auch unter restriktiven Netzwerkumgebungen hinter Firewalls oder bei Verwendung von Network Address Translation (NAT). 
Unter den Privatpersonen ist die Internet-Telefonie besonders bei den 18- bis 24-J�hrigen beliebt. In dieser Gruppe verwende fast jeder Vierte seinen Breitbandanschluss, um mit Freunden und Bekannten zu sprechen, das zeigten aktuelle Studien der FGW Online und des E-Business-Watch \cite{E-Business:08}. 

Parallel zur Entwicklung des VoIP Markets zeigen gleich mehrere gro�e Studien belegen die zunehmende Marktrelevanz von Computerspielen, die vor allem bei den gleichen Zielgruppen beliebt sind. So zeigt die Studie von Ernst\&Young "Digitale Spiele in Deutschland" \cite{ErnstYoung:07}, dass der Umsatz mit Konsolen-Spielen und PC-Spielen 2007 um 21 Prozent auf 2,14 Milliarden Euro, allein in Deutschland klettert. 2006 lag der Wert noch bei 1,77 Milliarden Euro, 2005 bei 1,57 Milliarden. Damit wird im laufenden Jahr erstmals die Marke von 2 Milliarden Euro erreicht. Ebenfalls erwartet eine die Studie German Entertainment and Media Outlook 2007-2011 von PriceWaterhouseCoopers (PwC) \cite{pwc:07}, dass in Deutschland im Jahr 2007 erstmals mehr Geld f�r Computer- und Videospiele als f�r Musik ausgegeben wird. Bis 2011 k�nnte der Umsatz der Ausgaben f�r Online- und Mobile-Games um j�hrlich 6,6 Prozent auf gut zwei Milliarden Euro wachsen. Weltweit wird der Umsatz der Unterhaltungs- und Medienindustrie bis 2010 auf 1,8 Billionen US-Dollar steigen. 

\section{Von "`Tennis for Two"' bis zur Kommunikationsplattform}

Der hohen Marktrelevanz von Computerspielen haben sie vor allem ihrem stetiger Wandel vom Computerspiel als akademisches, nichtkommerzielles Nebenprodukt zu einer ausgereiften kommerziellen Spiele- und Kommunikationsplatform zu verdanken: Die akademischen Anf�nge finden sich im Jahr 1958, wo der Physiker Willy Higinbotham ein Spiel das sich "Tennis for Two" nannte entwickelte, welches auf einem Osziloskop gespielt wurde. Diesem sollte im Jahre 1961 das Spiel "`Spacewar"' folgen, das an der Universit�t Stanford von Steve Russell entwickelt wurde \cite{CHM}. Erst weitere Jahre danach entstanden die ersten kommerziellen Computerspiele, so wurde "`Pong"' das im Jahre 1972 auf dem Atari System ver�ffentlicht wurde zur Sensation und zum ersten kommerziellen Erfolg.Als kommerzielles Produkte wurden Computerspiele in den 80er Jahren wurden haupts�chlich in Spielhallen gespielt oder von Spielern auf einfachen Konsolen zu Hause. Dabei bestanden bekannte Vertretern der ersten Generation wie Pacman oder Space Invaders einzig aus einem Einzelspieler Modus mit dem Ziel des Erreichens der H�chstpunktzahl. 
Ende der 80er Jahre verlagerte sich das Zentrum der Spieleindustrie nach Japan wo die ersten Generationen von Konsolenspielen gro�en kommerziellen Erfolg erreichten. Ein Hauptfaktor f�r war vor allem die M�glichkeit lokal mit mehreren Spielern zeitgleich auf dem Spielesystem zu spielen. Mitte der 90 er Jahre wurde der Personal Computer als Spieleplatform entdeckt. Vor allem durch das Konzept der Tastatur und Maus wurden neue Spielegenres erm�glicht, die durch durch eine "`point and click "' Steuerung mit Hilfe der Maus durch ihre Bedienfreundlichkeit �berzeugten.  Zwar hatten sich Tastatur und Maus sich als sehr ergonomisches "`Single-Player-Interface"' behauptet, erm�glichten jedoch im Vergleich zu Spielekonsolen nur sehr eingeschr�nkte lokale Mehrspieler M�glichkeiten. Durch den fr�hen Zugang des Personal Computers zu lokalen und globalen Netzwerken Ende der 90er Jahre bestand die M�glichkeit beliebteste Spielegenres mit Spielern auf der ganzen Welt zu spielen. Erste erfolgreiche kommerzielle Anbieter konnten 1997 bereits bis zu 750 000 Abonnenten. \citep{uo08}. Die Tendenz zum Spiel als Kommunikationsplattform wird heutzutage durch etliche gro�e Anbieter best�tigt, die Spiele mit online Mehrspieler Funktionalit�t anbieten, bei denen bis zu 10 Millionen Spieler in Spielergruppen miteinander spielen und kommunizieren \citep{wow2.2}. Aus akademischen Anf�ngen, �ber das streben des einzelnen Streben nach der H�chstpunktzahl ist heutzutage eine soziale Erfahrung geworden, die vor allem durch durch Kommunikation erst erm�glicht wird. \citep{triggerhappybuch}

\section{Kommunikation }

\subsection{Modell der Kommunikation}

Das Wort "`Kommunikation"' kann sehr umfangreich definiert werden. Die Ans�tze unterscheiden sich grunds�tzlich anhand der Frage, ob die Teilnehmer einer Kommunikation ausschlie�lich als Menschen bestimmt werden, oder allgemeiner als Lebewesen, zu denen dann auch die Tiere gez�hlt werden, oder ob die Teilnehmer einer Kommunikation als technische Ger�te (z.B. Computer) angesehen werden. 

Das allgemeine Modell von Shannon und Weaver  \citep{shannon48}, den Begr�ndern der Informationstheorie, besteht aus den folgenden Komponenten:
\begin{itemize}
	\item Einer Informationsquelle die eine Nachricht produziert.
	\item Einem Transmitter aus der Nachricht ein Signal generiert das durch einen Kanal geschickt werden kann. 
	\item Einen Kanal der das Medium bildet �ber welches das Signal �bertragen wird. 
	\item Einen Empf�nger, der das Signal wieder zu einer Nachricht transformiert. 
	\item Einem Ziel, einer Person oder Maschine, an die die Nachricht gesendet wurde.
\end{itemize}

Obwohl es nicht prim�r das Ziel des Modell war menschliche Kommunikation zu beschreiben, sondern die Daten�bertragung mithilfe technischer Faktoren optimieren, reicht dieses Modell vollkommen aus um die auftretenden Probleme aufzuzeigen:

\begin{itemize}
	\item Stufe 1: Wie genau k�nnen die Symbole der Kommunikation �bertragen werden? (Technisches Problem)
	\item Stufe 2: Wie pr�zise entsprechen die �bertragenen Symbole der urspr�nglich gew�nschten Bedeutung (Semantisches Problem)
	\item Stufe 3: Wie effektiv wird die empfangene Nachricht in eine gew�nschte Handlung umgesetzt (Effektivit�ts Problem)
\end{itemize}

Das technische Problem befasst sich mit der Genauigkeit der �bertragenen Symbole vom Sender zum Empf�nger eines Signals wie Schrift oder Sprache. Das Semantische Problem befasst sich mit der Interpretation der Bedeutung durch den Empf�nger im Vergleich zur urspr�nglichen Interpretation. So kann auch bei der einfachsten Nachricht eine Ver�nderung des Inhaltes oder der Bedeutung entstehen, falls die Semantik nicht eingehalten wird. Das Effektivit�ts Problem befasst sich mit der Aussicht, dass der Empf�nger tats�chlich das gew�nschte Verhalten einnimmt. 

Das bestehende Modell wurde sp�ter von Willbur Schramm \citep{schramm54} unter anderem um das Feedback erg�nzt, dass er als unentbehrlich bei der Kommunikation betrachtete. F�r Schramm gibt es nicht nur das Feedback durch eine Antwort oder durch Signale wie Kopfnicken, sondern es gibt auch eine Art von Eigenfeedback als Antwort zur selbst gesendeten Nachricht. Dabei ist eine kurze Antwortzeit und somit ein schnelles Feedback essentiell f�r eine erfolgreiche Kommunikation. So unterscheiden sich die schriftliche und verbale Kommunikation bei ihrer Antwortzeit, die bei der einen fast null ist bei der anderen sehr lange dauern kann. Je k�rzer die Antwortzeit, desto nat�rlicher erscheint eine Kommunikation.

\subsection{Interpersonelle vs. Massenkommunikation}

Die interpersonelle Kommunikation beinhaltet mindestens zwei Kommunizierende die beabsichtigt miteinander kommunizieren wobei beide sind Sender und Empf�nger zugleich sind. Dabei werden immer immer abwechselnd einzelne Nachrichten ausgetauscht. Unser allt�glicher Umgang miteinander im Alltag ist oft als interpersonelle Kommunikation zu verstehen. 

Bei einer Massenkommunikation kommuniziert ein Sender an viele verschiedene Empf�nger. Damit ist eine �ffentliche, indirekte, einseitige, technische Verbreitung von professionalisierter, strukturell und funktional 
ausdifferenzierter Kommunikation an ein disperses Publikum zu verstehen \citep{maletzke98}. 
Im Alltag findet eine Massenkommunikation Veranstaltungen statt, bei denen ein Redner vor einem Publikum redet, wobei das Fernsehen und Radio auch als Massenkommunikation verstanden wird. 

\subsection{Interaktivit�t und Lebhaftigkeit}
Im Bereich der Sozialwissenschaften spricht man von Interaktivit�t, wenn zwei Individuen miteinander im Kontakt sind und sich in ihren wechselseitigen Handlungen gegenseitig beeinflussen. Interaktivit�t oder auch die Wechselwirkung von Handlungen unterschiedlicher Personen aufeinander, kann unmittelbar zwischen Personen  oder vermittelt durch Medien wie Telefon, E-Mail oder Chat geschehen.

Interaktivit�t erh�ht die Qualit�t mit der wir das Kommunizierte verstehen. \citep{jensen98} Liest man Texte deren Sachverhalt nicht Grafiken weiter veranschaulicht wird, kann es vorkommen das man Teile des Erkl�rten nicht versteht, l�sst man sich dagegen den Stoff von einem Experten vortragen und kann z.B. durch Fragen Interaktiv den Stoff diskutieren so hat man oft ein besseres Verst�ndnis der Materie. 

Eng verbunden mit der Interaktivit�t ist die Lebhaftigkeit des Kommunizierten. So sind B�cher und Emails nicht besonders Lebhaft, weil man keine weitere Information wie ein Bild oder einen Ton mit dem gelesenen verkn�pft. Ein anderes Beispiel ist das Fernsehen, das sehr Lebhaft ist aber nicht besonders Interaktiv. In der Medienpsychologie wird die Lebhaftigkeit im Bezug zur Interaktivit�t als Begriff der Media-Richness-Theorie definiert. \citep{rice92}

% Grafik Interaktivit�t vs. Lebhaftigkeit
% Sekund�rliteratur MP2

Den Reichtum ('Richness') eines Mediums kann man daran messen, wie unmittelbar das Feedback ist, wie viele 
Kan�le wie viele Hinweise geben, wie pers�nlich die Kommunikation ist und wie vielf�ltig die vermittelte Sprache ist. Die Verwendung von besser geeigneten Medien f�hrt zu h�herer Effektivit�t der Aufgabenerf�llung. Reichwald  \citep{reichswald98} entwickeln daraus ein Media-Richness-Modell f�r die Telekooperation 
(vgl. Abbildung). In Abh�ngigkeit davon, wie mehrdeutig die Telekooperationsaufgabe ist, sind andere 
Medien zu bevorzugen. Dabei ist es nicht so, da� reiche Medien per se 'besser' geeignet sind und arme Medien schlechter. Vielmehr gibt es einen Bereich effektiver Kommunikation. Die Wahl zu reicher Medien f�hrt zu einer �berkomplizierung ('Overcomplication') der Situation. Anstatt Fakten zu suchen, werden die Teilnehmer 
durch den Reichtum des Mediums abgelenkt; es wird interpretiert und m�glicherweise Mehrdeutigkeit k�nstlich erzeugt. Die Verwendung zu armer Medien f�hrt zu einer zu starken Vereinfachung ('Oversimplification'): Das Medium eignet sich nur f�r die Informationssuche, obwohl ein gemeinsames Verst�ndnis durch gemeinsame Interpretation gefragt ist. Wegen mangelnden Feedbacks und Unpers�nlichkeit des Mediums kann nicht gemeinsam interpretiert werden. 

\subsection{Nonverbale Kommunikation}

Als nonverbale Kommunikation (deutsch Verst�ndigung ohne Worte) wird der Teil der Kommunikation des Menschen bezeichnet, der nicht mittels einer gesprochenen, geb�rdeten oder geschriebenen Sprache erfolgt, sondern durch nichtlinguistische Mittel wie K�rperhaltung, Gesten, Mimik, stattfindet. Die Kinesik ist die Wissenschaft, die sich mit der nichtsprachlichen Verst�ndigung befasst.
Dabei wird bei der Nonverbalen Kommunikation zwischen der Unbewussten (z.B. Aufnahme von Pheromonen) , Teilbewussten (z.B. Schweissbildung) und der Bewussten nonverbalen Kommunikation (z.B. L�cheln, Gestik) unterschieden. 

Dabei wird der Raum in drei Distanzzonen eingeteilt \citep{hall05}: Intime Distannz(50cm), Nahdistanz(1-3m), �ffentliche Distanz(mehr als 3m). Die Nahdistanz oder Soziale Zone hat sich auf Grund der mittleren Reichweite normal gesprochener Sprache gebildet. Hier kann von lebhafter Kommunikation ausgegangen werden, die andererseits nicht unmittelbar bedrohlich (handgreiflich) werden kann. In der �ffentlichen Distanz bewegen wir uns relativ sicher. Die "Obacht" l�sst nach, da potenzielle Gegner aus dem Umfeld eine gewisse Distanz zu �berbr�cken haben, bis sie uns erreichen. Verbale Kommunikation ist mit erhobener Stimme m�glich, oft werden Gesten zur Verst�ndigung eingesetzt. Distanzzonen finden sich bisher nicht in der Text-, Sprach- und Telekooperation, da die Kollaboration nicht an eine Repres�ntation des eigenen Ichs in der virtuellen Umsetzung gekoppelt ist. 

\section{Kommunikation in Computerspielen}

Mehrspieler-Computerspiele spielen in detaillierten, realistischen virtuellen 3D Welten durch die man mittels einer Spielfigur navigiert. In diesen Welten, kommunizieren Spieler miteinander aus verschiedenen Gr�nden wie z.B. um die Stragegie zu besprechen, Hilfe zu rufen, die Leistung des anderen zu bewerten oder um einfach nur zu plaudern. Dabei sind meistens in einem Spiel mehrere Wege der Kommunikation m�glich.

\subsection{Spielegenre}

Computerspiele lassen sich in 5 Hauptkategorien klassifizieren, wobei die Zuordnung keinen strengen Richtlinien folgt und ein Spiel auch mehreren Genres angeh�ren kann.

\begin{itemize}
	\item Actionspiele: Spiele in denen die Spielmechanik �berwiegend die Geschicklichkeit und Reaktionsschnelligkeit des Spielers fordert. Dies geht in der Regel mit einer starken Betonung des Echtzeit-Aspekts einher. In den meisten Actionspielen lenkt der Spieler eine einzelne Spielfigur. Dieses Genre wird dominiert durch Ego- oder First-Person-Shooter.
	
	\item Rollenspiele: Spiele die sich durch eine komplexe Handlung in einer erdachten oder adaptierten Welt verschiedenster kultureller, sozialer und zeitlicher Hintergr�nde auszeichnen. Sie  bieten die M�glichkeit einen oder mehrere Charaktere zu erschaffen, auszustatten und durch im Spielverlauf gesammelter Erfahrung sich entwickeln zu lassen. 
	Eine immer st�rker vertretene Klasse dieses Genres sind sog. MMORPGs (Massive-Multiplayer-Online-Roleplaying-Games) bei denen einen Schwerpunkt auf die Interaktion zwischen m�glichst vieln Spielern und Spielergruppen gelegt wird. Der Echtzeit-Aspekt spielt hier im Gegensatz zu klassischen Rollenspielen eine wichtige Rolle, da Aufgaben und R�tseln meist nur durch eine kollektive synchronisierte Zusammenarbeit mehrerer Spielergruppen gel�st werden k�nnenn. 
	
	\item Strategiespiele: Spiele dessen Bew�ltigung vor allem strategisches oder taktisches Geschick erfordert. Dabei �bernimmt der Computer entweder die Rolle eines Gegenspielers oder er bietet eine Plattform, auf der mehrere Spieler mit- bzw. gegeneinander spielen k�nnen. Stretegiespiele k�nnen sowohl Runden- als auch Echtzeitbasiert sein. 
	
	\item Simulationsspiele: Spiele bei denen die Durchf�hrung einer Simulation mit Hilfe eines Computers im Mittelpunkt steht. Dabei kann der Spieler Parameter der Simulation ver�ndern und deren Auswirkung auf das Simulationsmodell ausprobieren. Gegenst�nde von Simulationen k�nnen Fahrzeuge, Flugzeuge, Sportarten als auch Wirtschaftssysteme sein. Wegen des oft betr�chtlichen zeitlichen Aufwands bis zum Durchlauf einer kompletten Simulation wird dieses Spielgenre haupts�chlich im als Einspieler-Spiel gespielt. 
		
	\item Puzzlespiele: Spiele deren prim�res Ziel ist L�sungen f�r komplexe Ausgangsfragestellungen zu erarbeiten. Dabei sind vor allem im Vordergrund die Effizienz und Optimalit�t der L�sung sowie die ben�tigte Zeit. Die Zeitliche Spanne eines Spiels kann wenige Sekunden als auch mehrere Wochen betragen. Zeitlich begrenzte Puzzlespiele werden oft als Echtzeit Spiele gegeneinander gespielt. 
	
\end{itemize}

\subsection{Textkommunikation in Spielen}
Spieler k�nnen sich �ber Tastatur unterhalten, wobei die ausgetauschten Nachrichten bei Rollenspielen �ber dem Kopf des Avatars angezeigt werden oder bei First-Person-Shootern als Lauftext kurz eingeblendet werden. In Strategiespielen werden die Nachrichten nur an Spieler der gleichen Seite �bertragen. Generell spielt Textkommunikation vor allem bei Rollen- und Strategiespielen eine besondere Rolle, da sie essentieller Teil des Spielens ist. Bei First-Person-Shootern dagegen spielt Text-Kommunikation nur eine Nebenrolle, da diese Spiele bereits viele Tastatureingaben ben�tigen um die Spielfigur zu steuern. Da es unm�glich ist die Figur zu steuern und per Text zu kommunizieren, werden oft nur kurze Nachrichten zum Zweck der Koordination ausgetauscht. 

\subsection{Non-verbale Kommunikation in Spielen} 
Eine weitere Form der Kommunikation kann das gezielte Steuern des eigenen Avatars darstellen, indem man diesen in verschiedene Bewegungsabl�ufe bringen kann um so z.B. Freude, durch Springen und Armheben der Spielfigur zu kommunizieren. Vor allem in Avatarbasierten Action und Rollenspielen Spielt die Animation des Avatars eine wichtige Rolle, da sie essentielle Informationen �ber den Zustand und Absichten des Mitspielers preisgibt. 

\subsection{Sprachkommunikation in Spielen}
Spieler k�nnen sich mittels eines Mikrofons und Kopfh�rer unterhalten, wobei die ins Mikrofon gesprochene Nachricht an alle Teilnehmer des Empfangskanals gesendet wird. Sprachkommunikation wird mittels zus�tzlicher Sprachserver erm�glicht, auf denen sich Mitspieler zun�chst vor Spielbeginn einloggen und einen gemeinsamen Kanal beitreten. Vor allem Actionspiele bed�rfen eines hohen Anteils an Sprachkommunikation, um im Mehrspieler-Modus, die fehlende Textkommunikation zu kompensieren. Strategie-, Simulations- und Puzzlespiele verzichten auf die Sprachkommunikation wegen eines Rundenbasierten Spielmodus oder dem fehlendem Mehrspielercharakters des Spiels. 

Rollenspiele nehmen eine Sonderl�sung ein, da einerseits in Echtzeit-Mehrspieler-Spielen Sprachkommunikation bei befreundeten Spielegruppen zunehmend genutzt wird, jedoch bei noch fremden Mitspielern wegen fehlender direkter Implementierung der Sprachkommunikation ins Spiel nicht m�glich ist. Eine durchgehend in alle Spielegenre integrierte Sprachkommunikation bietet als einziges System bisher XBox-Live Konsole, die aufgrund Fehlender Tastatureingabe einzig die Kommunikation �ber das Mikrofon und Kopfh�rer erm�glicht. 


\section{Vor und Nachteile der Text-Kommunikation}
Untersucht man die Textkommunikation bez�glich ihrer Interaktionsm�glichkeiten so ist festzustellen, dass sich Verbal-Verhalten gut abbilden l�sst, weil die Schriftform die Kinesik und Stimm-F�hrung komplett ausblenden kann. Einfach gehaltene Textnachrichten dienen oft der Koordination im Spiel und werden von den Spielern nur im Hintergrund des Spiels gelesen. Lange Interpersonelle Kommunikation findet oft nicht im Spiel statt sondern au�erhalb in Chatrooms vor der Spielesession oder nach danach. Die Massenkommunikation �berwiegt meist, da eine gezielte Addressierung der Mitspieler oft nicht m�glich oder aus Effizienzgr�nden nicht gewollt ist. Betrachtet Textkommunikation unter den Problemkategorien der Kommunikation so kann man feststellen, das das technische Problem der reinen Daten�bertragung bereits durch verl�ssliche zugrundeliegende Netzwerkschichten minimiert wird, dagegen das Semantische- und Effektivit�tsProblem oft unabh�ngig vom Spielegenre weiterhin ein offenes Thema sind. So ist sind mit bei der Texkommunikation spieler durch ihre Schreibgeschwindigkeit limitiert und gerade bei Echtzeit basierten Spielen bereits durch das Spielgeschehen so eingenommen,dass eine detaillierte semantisch korrekte Kommunikation oft unm�glich ist. Durch fehlende Semantik und eine lange Reaktionszeit sowie fehlendes Feedback verschlechtert sich ingsgesamt auch die die Effektivit�t der Kommunikation. Gerade bei den Hauptgenres des Mehrspielerspiels ist ein Spagat zwischen dem Eintauchen in die Spielewelt und der daraus resultierenden konstantem Spielerinteraktion mit dem Spiel und der Spielerinteraktion der Spieler untereinander nicht l�sbar. So ergeben sich durchgehend Feedbackzeiten bei Antworten auf Textnachrichten von mehr als 2 Sekunden. Die Interaktivit�t solcher Konversationen wird dadurch stark Eingeschr�nkt, weil das gewohnte Tempo nicht eingehalten werden kann und Verz�gerungen an der Tagesordnung sind. Die Reduzierung der zwischenmenschlichen Kommunikation auf die reine Textkommunikation f�hrt zu einer geringen Lebhaftigkeit der erlebten Kommunikation. Die Schlichtheit einer Textkommunikation kann oft mit einer grafisch opulenten Darbeitung der Spielewelt nicht mithalten und verk�mmert so zu einer unbedeutenden Nebenrolle im Spielgef�ge. 

Allerdings bietet die Text-Kommunikation laut \citep{Turkle and Reid} gerade trotz ihrer technischen Einfachheit auch neue M�glichkeiten. Was passiert wenn Mitspieler sich anders darstellen als sie sich in realen Situationen darstellen w�rden? Studien haben gezeigt, dass Menschen neue parallele Identit�ten erschaffen k�nnen um f�r sich mit Sexualit�t, Rasse, Geschlecht und Macht zu experimentieren. Solche Identit�ten k�nnen aktiv online gelebt werden, ohne dass das Gegen�ber die Wahre Identit�t des Spielers erf�hrt, da durch die Textkommunikation essentielle Details ausgeblendet werden k�nnen, und gerade das ist das was sie immer so attraktiv macht. 

\section{Vor und Nachteile der Nonverbalen-Kommunikation}
Untersucht man die Nonverbale Kommunikation bez�glich der von uns definierten Kommunikationseigenschaften so sind viele unserer Kriterien nur bedingt einsetzbar. Bez�glich der Interaktionsm�glichkeiten bietet sie keinerlei Verbal-Verhalten und bietet auch keine M�glichkeiten einer Stimmf�hrung kann jedoch den Apekt der Kinesik abbilden. So sind gerade in Rollenspielen mehrere vordefinierte Animationsm�glichkeiten der eigenen Spielfigur m�glich um die Text-Kommunikation durch die Fehlende Kinesik zu erg�nzen. Man muss jedoch betonen, dass vorhandene Systeme die die Mimik eines zwischenmenschlichen Gespr�ches nur in einem sehr groben Detail abbilden k�nnen, weil die zugrundeliegenden Modelle das komplexe Zusammenspiel aller Bewegungen nur rudiment�r abbilden kann. Nonverbale Kommunikation im Spiel kann sowohl als Massen- als auf Interpersonelle-Kommunikation verstanden werden, hier ist lediglich der Sichtradius der Mitspieler der ausschlaggebende Faktor. Unter dem Spakt der Problemkategorien sind f�hrt vor allem der gro�e Interpretationsspielraum einer Nonverbalen Kommunikation zu gro�en semantischen Problemen und somit verbundenen Effizienzproblemen. Die M�glichkeit der Interaktivit�t auf nonverbalem Level wird wegen ihrer kaum vorhandenen Semantik kaum genutzt trotzdem tr�gt die nonverbale Kommunikation stark zur Erh�hung der Lebhaftigkeit der Kommunikation dar. So liegt der Hauptvorteil der Nonverbalen Kommunikation in Spielen bisher vor allem in der Erh�hung der Lebhaftigkeit der Textkommunikation und Aspekte der Kinesik dagegen sind vor allem durch komplizierte filigrane menschliche Bewegungsvorg�nge noch nicht in der Detailstufe abbildbar und Steuerbar, als das sie die Semantische Ebene der Text-Kommunikation erweitern k�nnten. 

\section{Vor und Nachteile der Sprachkommunikation}
Die Sprachkommunikation ist bei der zwischenmenschlichen Interaktion die dominierende Kommunikationsart. In der zwischenmenschlichen Interaktion in Computerspielen jedoch ist sie noch kein Integraler Bestandteil geworden. Trotzdem soll eine Untersuchung bez�glich der definierten Kenngr��en zeigen, wie weit sie mit den  bereits evaluierten Kommunikationsmethoden vergleichbar ist. Bez�glich der Interaktionsm�glichkeiten l�sst sich feststellen, dass sowohl Verbal-Verhalten als auch die Stimmf�hrung ohne weiters m�glich sind. Das Mittel der Kinesik ist jedoch in der reinen Sprachkommunikation nicht m�glich, da keine �bertragung der eigenen Bewegungen m�glich ist. Die Sprachkommunikation wird in Computerspielen aufgrund technischer Einschr�nkungen meistens nur als Massenkommunikation genutzt bei der sich bis zu mehrere Spieler einen gemeinsamen Kanal teilen. Da bereits eine Kollision von 2 Sprach�bertragungen ausreicht um das gesprochene nicht mehr verstehen zu k�nnen, unterliegen werden solche Kan�le stark reglementiert und die Sprachkommunikation von einzelnen Spielern zur Koordination der Gruppe benutzt. Interpersonelle gespr�che auf diesen Kan�len sind somit fast ausgeschlossen. Obwohl die Interpersonelle Sprachkommunikation im zwischenmenschlichen Bereich im Alltag als fest etabliert ist, und sowohl Telefone, Mobiltelefone und Voice-Over-IP L�sungen sehr verbreitet sind, findet in Spielen eine personalisierte Nutzung des Mediums aus oben genannten Gr�nden nicht Statt. Das gro�e Potenzial von Spielen als interpersonelles Kommunikationsmedium wird somit nicht voll ausgenutzt. Der Gr�nde f�r eine solches Vers�umniss liegen vor allem im technischen Bereich. 
Untersucht man die Problemkategorien der Sprachkommunikation, so ist im Gegensatz zur Textkommunikation ein hoher technischer Aufwand notwendig um eine funktionierende Sprachkommmunikation aufrecht zu erhalten. So k�nnen schon auf der niedrigsten Stufe technische Probleme auftreten die das Gesprochene f�r das Gegen�ber unverst�ndlich machen. Schon auch die Einstellungen an Mikrofon und Lautsprechern oder Hintergrundger�usche k�nnen zu einer starken Beintr�chtigung der Kommunikation f�hren. Trotz Probleme im technischen Bereich hat die Sprachkommunikation geringere Probleme bei den h�hren Stufen in der Semantik und Effizienz. Da kein Medienbruch wie bei der Textkommunikation stattfindet, wird das semantische Problem auf ein das gleiche semantisches Problem reduziert, dass wir auch in der zwischenmenschlichen Kommunikation in unserem Alltag finden. Durch eine starke Verk�rzung der Feedback Zeiten bei der Sprachkommunikation gekoppelt mit einer einfach verst�ndlichen Semantik kommt es zu einer starken Erh�hung der Effizienz der Kommunikation. So k�nnen Spieler in sekundenbruchteilen auf �nderungen im Spielfluss reagieren und andere Spieler warnen oder Hilfe anfordern. 

Da die Sprachkommunikation in Echtzeit verl�uft, gilt die Wiedergabe eines Sprachsignals am Ziel asl qualitativ schlecht, wenn sie einem zu gro�en Zeitverzug erfolgt. F�r die Ende-zu-Ende-Verz�gerung $T_{EE}$(End-to-End-Delay) des Sprachsignals werden daher Grenzwerte gesetzt. Nach dem ITU-T-Dokument G.114 wird die VoIP-Qualit�t wie flogt klassifiziert:
\begin{itemize}
	\item $T_{EE}$ kleiner als 150ms: akzeptabel f�r alle Benutzer,
	\item $T_{EE}$ zwischen 150ms und 300ms: akzeptabel, aber mit Einschr�nkungen (nicht f�r empfindliche Benutzer), 
	\item $T_{EE}$ gr��er als 300ms: nicht akzeptabel. 
\end{itemize}

Zwar ist bei der Textkommunikation der Zeitverzug weitaus� h�her, schon bedingt durch den limitierenden Einfluss der pers�nlichen Schreibgeschwindigkeit der Gespr�chspartner, trotzdem werden diese Antwortzeiten nicht als Hinderniss wahrgenommen. Im Gegensatz zur Sprachkommunikation findet hier eine abstrahierte asynchrone Kommnunikation statt, w�hrend bei der Sprachkommnuikation jegliche Asynchronit�t als eine Unterbrechnung des Kommunikationsflusses gesehen wird. Bei den Kenngr��en der Interaktivit�t dominiert die Sprachkommunikation jedoch durch die direkte intuitive R�ckkanalm�glichkeit die vergleichenen Kommunikationsarten. Die Lebhaftigkeit von gesprochener Kommunikation wird weithin als hoch angesehen und so wird einzig die Video-Kommunikation und ein "`Face-to-Face"' Gespr�ch als noch lebhafter wahrgenommen. Da der Spieler mittels der Sprachkommunikation noch tiefer in das Spielgeschehen eintaucht wird das erlebte noch lebhafter wahrgenommen. Trotz ihrer technischen Einschr�nkungen vor allem in Bezug auf die interpersonelle Kommunikation in Spielen ist die Sprachkommunikation in weiten Bereichen den anderen Kommunikationsarten stark �berlegen. Da Sprache auch unsere Kommunikation im Alltag bestimmt, liegt es Nahe, dass dieses Medium auch die vorherrschende Kommunikationsmethode in Computerspielen werden kann. Dar�ber hinaus bietet sie im Vergleich zur reinen Textkommunikation noch weitere Vorteile die ihr als Alleinstellungsmerkmal dienen:

\subsection{Potenzielle Alleinstellungsmerkmale der Sprachkommunikation}

\subsubsection{Integration von Sprachkommunikation in das Computerspiel}
Obwohl der Status Quo der Sprachkommunikation heutzutage immernoch nur durch den Einsatz von zus�tzlicher Software erm�glicht werden kann, ist es absehbar dass die Sprachkommunikation direkt in das Spiel integriert und verzahnt werden kann. Durch so eine Verzahnung sind neue M�glichkeiten M�glich Sprache als Teil des Spielens zu verstehen und Spiele zu Kommunikationsplattformen aufzubauen. Einige Vorreiter im Kosolenbereich bieten bereit einen Aboservice an wie z.B. die XBox Live, das in k�rzester Zeit zu einem der profiliertesten und popl�rsten Sprachkommunikations Spielportal wurde. Obowhl das System mit vielen Kinderkrankheiten zu k�mpfen hat ist die Zukunft der Sprachkommunikation als ausgereiftes Kommunikationsmedium absehbar. 

\subsubsection{Erh�hung der Lebendigkeit}
Dadurch dass Spieler nicht mehr zwischen dem Spiel selbst und Texteingabefeldern hin und herschalten m�ssen, k�nnen sie sich vollkommen auf das Spielgeschehen konzentrieren. Ausgestattet mit einem Headset k�nnen sie so mit ihren Mitspielern in die Spielewelt eintauchen und so die Lebhaftigkeit enorm erh�hen. 

\subsubsection{Schnellere Reaktion}
Durch eine direkte Sprachkommunikation ist es viel einfacher f�r Mitspieler auf das Spielgeschehen zu reagieren. Vor allem bei den Hauptvertretern der Multispieler Spieler, wie den Ego-Shootern und Echtzeit-Rollenspielen ist die Reaktionszeit eine wichtige Komponente. Durch die Sprachkommunikation reduziert sich die Reaktionszeit im Vergleich zur Textkommunikation auf ein Minimum.  

\subsubsection{Spiel als Kommunikationsplattform}
Die weite Verbreitung von VoIP im privaten Bereich, hat zu einer Revolution unserer Kommunikationsgewohnheiten gef�hrt. So sind wir in der Lage Freunde direkt und kostenlos anzurufen. Durch eine Integration der bestehenden VoIP Standarts in Computerspiele w�ren wir in der Lage, dies auch direkt aus dem Computerspiel zu erledigen. Somit w�rde die Grenzen zwischen dem Spiel als Entertainment Platform und Spiel als Kommunikationsplattform nicht mehr existieren. So wird das finden von Mitspielern f�r ein Spiel bereits Teil des Spiels. --> PS3 

\subsubsection{Keine H�nde}
Obwohl es trivial ist, ben�tigt man im Gegensatz zur Textkommunikation keine H�nde. Der Spieler kann sich voll und ganz auf das Steuern des Avatars konzentrieren, ohne dass er das Spielgeschehen dauernd unterbrechen m�sste. Computerspiele und ihre KOmmunikation sind ebenfalls somit nicht mehr auf die Tastatur als Kommunikationsger�t angewiesen, und wie schon bei der XBox Live Konsole gezeigt, scheint die Einfachheit einer solchen Kommunikation auf weite Akzeptanz zu sto�en. 

\subsubsection{Erh�hter Realismus}
Obwohl es ausnahmen gibt, bei denen man die Textkommunikation als Mittel nutzen kann um seine Identit�t neu auszuleben, erh�ht die Sprachkommunikation den Realismus eines Spiels. Gerade durch die Stimmf�hrung sind kann die Sprachkommunikationen das Spielerlebniss zwischen zwei Spielern stark beinflussen. 

\subsubsection{Mobile Einsatzbereiche}
Ein untersch�tztes Nebenprodukt der Sprachkommunikation ist, dass man aufgrund nicht ben�tigter Tastatur auch bei mobilen Ger�ten die M�glichkeit hat miteinander zu Kommunizieren. So ist es vorstellbar auf tragbaren Konsolen Mittels VoIP und WLan Spiele miteinander zu spielen und gleichzeitig Miteinander zu telefonieren. 

\subsubsection{Talking with People from All Over}
Mittels VoIP ist es bereits heute m�glich und �blich mit Freunden und bekannten auf der ganzen welt kostenlos zu telefonieren. Obwohl soziale Netzwerke im Internet ein Spiegel der real existieren Netzwerke sind, bieten Spiele die M�glichkeit sich �ber die Methode des gemeinsamen Erlebens Freundschaften zu Entwickeln. Gerade im Spielebereich ist eine starke Bindung zu Spielergruppen �blich. Bisher findet Kommunikation innerhalb solcher Gruppen einzig in Textform statt, mit Sprachkommunikation k�nnen st�rkere Bindungen der Spieler untereinader erfolgen, und mittels einer Interpersonellen Sprachkommunikation im Spiel auch pers�nliche Themen diskutiert werden, w�hrend man gemeinsam seine Freizeit im Third Place verbringt.
--> Theory of third place
-->  Increased Talking Abilities --> Talk to anyone

\subsubsection{Kein Hin- und Her-Schalten zwischen Applikationen}
Obwohl Sprachkommunikation bereits seit einigen Jahren Teil von Computerspielen ist, wird sie oft nur vereinzelt von technisch versierten Benutzern genutzt. Oft ist es erforderlich gerade in zeitkritischen Spielen zwischen verschiedenen Applikationen hin und herzuschalten um mit neuen Teilnehmern des Spiels eine Sprachkommunikation aufzubauen. Durch eine Integration der Sprachkommunikation in das Spiel ist es auch technisch unterfahrenen Nutzern m�glich einfach und komfortabel immer mit Mitspielern Sprachverbindungen aufzubauen. 
--> H�rden beim Benutzen von Computern

\subsubsection{Einfache Koordination}
Will man mit mehreren Teammitgliedern Sprechen so sind gerade die in Spielen vorgesehenen Sammelphasen, die bis zu 10 Sekunden dauern optimal um mittels Sprachkommunikation die weitere Vorgehensweise zu besprechen. W�hrend bei der Textkommunikation diese kurze Zeit oft nur f�r unzureichende Strategische anweisungen ausreicht oder die Spielstrategie wegen unzul�nglicher Kommunikation nicht ge�ndert wird, kann mittels Sprachkommunikation eine weit aus bessere Koordination im Spiel erreicht werden. 

\subsubsection{Stimmverfremdungen}
Es is auch vorstellbar, dass man die eigene Stimme verfremden kann, um einen bestimmten Effekt zu erreichen der den Mitspieler noch weiter in das Spiel eintauchen l�sst. So k�nnte die Stimmlage an die Avatare angepasst werden oder eine d�stere Stimmlage erzeugt werden, um dadurch andere Spieler zu verschrecken. 

\subsubsection{Standardisierung}
Obwohl mehrere Sprachkommunikations L�sungen auf Basis von propri�teren Technologie existieren, ist es durchaus vorstellbar, dass die Standardisierung die im VoIP bereits weit gediegen ist, auch im Spielebereich Einsatz h�lt. So ist es vorstellbar, dass Sprachkommunikation auf einem offenen Protokoll wie z.B. SIP basieren kann und Anbieterl�sungen somit kompatibel werden.

\subsubsection{Sprache als Vergleichendes Kriterium}
Obwohl die Interaktion in einem Computerspiel im Rahmen der gegebenen M�glichkeiten passiert, und der Gewinn eines Spieles dadurch durch die Eigenen F�higkeiten entschieden wird, kann auch Sprache zu einer Spielentscheidenden F�higkeit werden, indem man seine Mitspieler verunsichert oder in die Irre f�hrt. Ebenso sind auch Spiele vorstellbar die einzig durch Sprache gesteuert und ausgetragen werden k�nnen. 

\subsubsection{Vertrauen}
\subsection{Nachteile von Audiokommunikation}
\subsection{�bertragung}
\subsection{Reichhaltigkeit}
\subsection{Verwandte Arbeiten}
\subsubsection{XBox Live und Konsolen}

\chapter{VoIP und SIP}
\section{Aufbau eines VoIP Systems}
Der einfache Aufbau eines funktionierenden VoIP Systems kann prinzipiell mit einer bestehenden IP-Infrastruktur und Softphones, die auf Rechners ausgef�hrt werden, funktionieren. Die Teilnehmer m�ssen jedoch bei einer solchen Konfiguration die IP-Adressen der gew�nschten Gespr�chspartner kennen um einen direkten Call auszuf�hren. Bei einer Adress�nderung der IP Adresse eines Gespr�chspartners, m�ssten jedoch alle Eintr�ge der Gespr�chspartner entsprechend abge�ndert werden, weil sich die Teilnehmer sonst nicht mehr erreichen k�nnten. Der gleiche Aufwand m�sste auch beim Hinzuf�gen weiterer Teilnehmer erfolgen. Ein solcher Aufbau wird daher als unflexibel angesehen. 
Im SIP-Protokoll wird dieses Problem von Proxys �bernommen, die die Aufgabe einer Vermittlungsstelle einnehmen. Ein Proxy kann einen Location Service enthalten, der ankommende SIP-Pakete an das Telefon des angerufenen Teilnehmers weiterleitet oder an einen anderen Proxy weiterleitet, falls sich der Benuutzer au�erhalb des eigenen Adressbereichs befindet. Zus�tzlich k�nnen Konferencen, Voice-Mail und Anrufweiterleitungen zentral verwaltet werden. 
Da die klassische Telefonie weiterhin neben VoIP Bestand hat, wird versucht mit zus�tzlichen Komponenten, den Gateways, eine Verbindung zwischen den beiden Telefonsystemen herzustellen. 

\subsection{Protokolle zur Echtzeitkommunikation}
Voice-over-IP beinhaltet unterschiedliche Protokolle. Zur reinen Signalisierung werden haupts�chlich das SessionInitiation Protocol (SIP) \cite{needed} bzw. Protokolle aus der H.323-Protokollfamilie verwendet. Bei VoIP sing generell folgende Klassen der Protokolle zu unterscheiden:
\begin{itemize}
	\item Protokolle f�r die Sprach�bermittlung: Da die Sprachkommunikation in Echtzeit verl�uft, sind spezielle Protokolle f�r die �bermittlung der Sprache �ber IP-Netze n�tig.
	\item Signalisierungsprotokolle: Es handelt sich hier um Protokolle f�r den Auf- und Abbau von Verbindungen zwischen zwei IP-Telefonen.
	\item Diese Protokolle sind n�tig, um die klassischen TK-Systeme und -Netze, wie z.B. das ISDN mit den IP-Netzen mit Hilfe von verschiedenen Gateways zu integrieren. 
\end{itemize}

Vorg�nge wie die Vermittlung von Gespr�chen, der Rufaufbau oder -abbau sowie die Aushandlung von Parametern, z.B. die Verwendung eines gemeinsamen Codecs, der entsprechenden Bitrate oder der maximal zul�ssigen Bandbreite werden in diesen Protokollen kontrolliert. Speziell f�r diese zwecke wurde das Session Description Protocol (SDP) \cite{needed} entwickelt um den Medienstrom zu kontrollieren. SIP basiert auf den gleichen Prinzipien, die man beim HTTP(HyperText Transfer Protocol) nutzt. 

F�r die Sprach�bertragung wird das Real-time Transport Protocol (RTP) \cite{needed} verwendet, das dem Empf�nger der Pakete erlaubt, trotz unzuverl�ssiger �bertragung �ber UDP \cite{needed}, einen Paketverlust festzustellen und die urspr�ngliche Reihenfolge der Daten wiederherzustellen. RTP ist ein Transprotokoll, ist also nicht nur f�r Sprache, sondern f�r alle Echtzeitmedien, wie Video verwendbar. Da RTP keine Best�tigungen verwendet, kann der Empf�nger dem Sender nicht mitteilen, ob �berhaupt btw. wie die Sprache in Form von IP-Paketen bei ihm ankommt. Das RTCP Protokoll gleicht diesen Nachteil aus: Mit Hilfe des RTP Control Protocols (RTCP) k�nnen zus�tzlich zu den reinen Mediendaten auch Kontrollinformationen wie die Anzahl der verlorenen Pakete oder der gemessene Jitter \cite {needed} der RTP-Verbindung �bertragen werden und so �bertragungsparameter angepasst werden. 

Neben den bisher erw�hnten Protokollen zur Hauptfunktionalit�t gibt es noch noch weitere, die sich mit allen angrenzenden Bereichen von VoIP besch�ftigen. So wird mit Hilfe des STUN (Simple Traversal of User Datagram Protocol Through Network Address Translators) erm�glicht seine IP-Adresse zu ermitteln, falls man sich hinter einem NAT Router befindet und die interne IP-Adresse nicht der externen entspricht. Dies ist vor allem notwendig, da die IP-Kontaktinformationen in SIP und SDP in der Anwendungsschicht �bertragen werden und somit ein Client ohne Kentniss richtigen Adresse falsche Informationen liefern w�rde. Ein weiteres Protokoll mit dem Ziel der �berbr�ckung eines NATs ist TURN(Traversal Using Relay NAT) \cite{needed}. Bei TURN bekommt ein Client von einem TURN-Server eine �ffentlich erreichbares IP-Adressen- und Port-Paar zugewiesen und leitet alle an diese �ffentliche Adresse ankommenden Pakete an den Client weiter. 

F�r die Ansteuerung von VoIP-Gateways stehen die Protokolle MGCP(Media Gateway Control Protocol) nund Megaco(Media Gateway Control) zur Verf�gung, werden im Rahmen der Arbeit nicht weiter erl�utert. 

Dieses Kapitel soll zun�chst den Aufbau eines VoIP Systems beschreiben um anschlie�end die Protokolle SIP/SDP, RTP und RTCP zu behandeln. Da SIP die �ltere H.323 Protokollfamilie in langfristig verdr�ngen wird, wird auf diese Protokollfamilie nicht behandelt. Im Abschluss wird ein �berblick �ber das propri�tere Protokoll von Skype gegeben, dessen Funktionsweise und Details zwar nicht offen sind, es jedoch eine Alternative zu den offenen Protokollen bietet. Da ein Reverse-Engineering dieses Protokolls au�erhalb des Aufgabenbereichs der Arbeit liegt.

\subsection{Einflussfaktoren auf die VoIP-Qualit�t}
Die wichtigsten QoS-Anforderungen, die VoIP an IP-Netze stellt, betreffen:
\begin{itemize}
	\item die Bandbreite von virtuellen Verbindungen zwischen IP-Telefonen,
	\item die Ende-zu-Ende-Verz�gerung des Sprachsignals (Delay),
	\item die Schankung der �bermittlungszeit (Jitter) und
	\item die Paketverlustrade (Packet Loss Rate)
\end{itemize}

Da bei VoIP die Sprache zuerst digitalisiert und dann enstsprechend codiert wird, ist dabei die Qualit�t des �bertragenen codierten Sprachsignals abh�ngig von der Bandbreite f�r die virtuelle Verbindung zwischen den IP-Telefonen. Ein wichtiger Faktor ist auch die Ende-zu-Ende-Verz�gerung des Sprachsignals. Darunter versteht man die Zeitspanne, die ein Sprachsignal vom Mund eines Sprechers bis zum Ohr eines H�rers ben�tigt. Die Ende-Zu-Ende-Verz�gerung entsteht vor allem durch die Zwischenspeicherung der IP-Pakete in den Routern, die sie auf ihren Wegen durch das Netz zu durchalufen haben. Jeder Router ben�tigt Zeit, um den Header im IP-Paket zu interpretieren und die entsprechende Routing-Entscheidung zu treffen. Trifft ein Paket unterwegs auf einen �berlasteten Route, muss es einige Zeit in der Warteschlnage vor der Leitung verbringen und wird im Extremfall sogar ganz verworfen. Eine gro�e Ende-zu-Ende-Verz�gerung beintr�chtigt den Charakter eines Telefongespr�ches stark. 

Da die einzelnen IP-Pakete auf einer Verbindung zwischen IP-Telefonen in der Regel auf unterschiedlichen Wegen �bermittelt werden, kann ihre �bermittlungszeit recht unterschiedlich sein. Die Schwankungen der �bermittlungszeit werden Jitter genannt. Die Synchronit�t bei VoIP l�sst jedoch kein Jitter zu. Um die Schwankungen auszugleichen, wird beim Empf�nger ein spezieller Puffer implementiert. Man bezeichnet einen derartigen Puffer als Jitter-Ausgleichspuffer(Playout Buffer)

Die Verluste von IP-Paketen mit Sprache w�hrend der �bermittlung mindern die Qualit�t ebenso. Die Anzahl der Paketverluste in einer bestimmten Zeitperiode wird mit dem Parameter Paketverlustrate angegeben. Paketverluste k�nnen durch �berlastete Router im IP-Netz oder auch durch einen "`schlecht"'  dimensionierten Jitter-Ausgleichspuffer entstehen. Im Gegensatz zu reinen Datenanwendungen, ist der Verlust eines IP-Pakets bei der Sprach�bertragung nciht besonders tragisch. Zu viele Paketverluste jedoch machen sich in einem Telefongespr�ch allerdings sehr st�rend als Unterbrechungen bemerkbar. 

\section{Session Initiation Protocol (SIP)}
\subsection{Aufbau}
\subsection{Dialoge, Transaktionen und Vorg�nge}
\subsection{Registrierung}
\subsection{Rufaufbau, Verbindung, Rufabbau}
\subsection{Forking}
\section{Session Description Protokoll}
\subsection{Aufbau}
\subsection{Aushandeln der Sitzungsparameter}
\section{Real-Time Transport Protokoll}
Speziell f�r die �bermittlung von Sprache, Audio und Video �ber IP-Netze wurde RTP(Real-time Transport Protocol) bei der IETF entwickelt (2003 RFC 2550). Einerseits stellt RTP ein Transportprotokoll f�r die Echtzeitmedien dar anderseits kann es als eine Anwendungsart oberhalb des verbindungslosen Protokolls UDP angesehen werden. 

--grafik s151

Die Wichtigsten Besonderheiten von RTP sind:

\begin{itemize}
	\item �bermittlung von Echtzeitmedien in RTP-Paketen: Echtzeitmedien werden in RTP als eine zusammenh�ngende Folge von RTP-Paketen �ber eine RTP-Session �bermittelt. 
	\item Garantie der Reihenfolge von RTP-Paketen: RTP nummeriert die �bertragenen Pakete mit Echtzeitmedien, sodass ihre richtige Reihenfolge am Ziel wiederhergestellt werden kann, falls sie durhc den Transport �ber das IP-Netz ver�nert wurde. 
	\item Grantie der Synchronit�t: RTP verfibt den �bertragenen Paketen einen Zeitstempel, sodass die gleichen Zeitabst�nde am Ziel wiederhergestellt werden k�nnen, die beim Absender bestanden. 
	\item Transport unterschiedlicher Formate: Es k�nnen unterschiedliche Formate wie Audio, Sprache und Video �bertragen werden, die in unterschiedlichen Proflies genau bezeichnet werden. 
\end{itemize}

\subsection{Aufbau von RTP-Paketen}
RTP �bermittelt ein Echtzeitmedium als Folge von RTP Paketen , die mit einem vorangestellten UDP-Header in den IP-Paketen transportiert werden. Jedes RTP-Paket enth�lt einen RTP-Header und einen Payload-Teil. Die Wesentlichen angaben im RTP-Header sind:
-- grafik  S153

\begin{itemize}
	\item PT: Payload Type: Es wird angegeben, um welches Format es sich beim transportierten Medium handelt, dh. nach welchem Verfahren codiert wurde. Die M�glichen Codierungsarten sind in RFC 3551 festgelegt. 
	\item Timestamp: Der Zeitstempel dient dazu, den Zeitpunkt der Generierung von Payload zu markieren. Der Zeitstempel ist n�tig, um die Schwankungen der �bertragungszeit (Jitter) von RTP-Paketen beim Empf�nger auszugleichen. 
	\item Sequence Number: Jedes RTP-Paket wird mit einer Sequenznummer versehen, die es dem Emfp�nger erlaubt, den Verlust von Paketen festzustellen, bzw. die Reihenfolge wiederherzustellen, falls sie in einer falschen Reihenfolge angekommen sind. 
	\item SSRC: Zur Identifikation der Quelle(Mikrofon, Kamera) dient der SSRC, so m�ssen zwei verschiedene Quellen unterschiedliche SSRCs haben, um f�r den Empf�nger unterscheidbar zu sein. 
	\item CSRC: Wird optional verwendet un enth�lt die "`Orginal"'-Quellen der Bitstr�me falls der SSRC von einem Mixer ver�ndert wurde.
\end{itemize}

\subsection{Payload-Typen}
Die PT Nummern dienen zur Identifikation der einzelnen Audio- und Video-Formate. Da der Nummernraum nicht komplett belegt ist k�nnen wietere nummern vergeben werden (dynamische PT-Nummern). Bei der Verwendung von dynamischen PT-Nummern wird die Kompatibilit�t zwischen Emfp�nger und Sender mit Hilfe des SDP Protokolls vereinbart. 

-- payload tabelle

\subsection{RTP Control Protokoll}
RTP wird mit RTCP so erg�nzt, dass INformation �ber den Verlauf der Kommunikation zwischen Sender und Empf�nger, insbesondere �ber die Qualit�t der �bertragung, ausgetauscht werden k�nnen. 

Die wichtigsten Aufgaben von RTCP sind ide �berwachung der �bertragungsqualit�t: Hierf�r werden zwischen Sender und Emfp�nger laufend Informationen �ber die Qualit�t der �bertragung mittels Sender- und Receiverreports ausgetauscht. Dies Erm�glicht es dem Sender, den von ihm generierten Bitstrom an die Netzbedingungen anzupassen und so Fehler einzugrenzen. 

\subsubsection{Typen und Struktur der RTCP-Pakete}
Beim RTCP werden folgende RTCP-Pakete als Nachrichten verwendet. 
\begin{itemize}
	\item Sender Report (SR)
	\item Receiver Report(RR)
	\item Source Description (SDES)
	\item Abmeldung (BYE)
	\item Applikationsspezifisches Paket (APP)
\end{itemize}

RTCP Pakete beginnen mit einem Header und k�nnen unabh�ngig von anderen bearbeitet werden. Die Reihenfolge der Pakete in einem IP-Paket kann beliebig sein. 
Das SR Paket enth�lt einen Zeitstempel gem�� der NTP (Network Time Prococol) und beschreibt die Qualit�t der �bermittlung aus Sicht des Senders. So kann z.B. dem Empf�nger die Sende-Datenrate mitgeteilt werden, damit dieser sich im Voraus auf die ankommende Datenmenge einstellen kann. 

% Grafik eines RTCP Pakets

In einem Sender Report sind immer enthalten:
\begin{itemize}
	\item NTP-Zeitstempel: Ein Zeitstempel gem�� der NTP Uhrzeit. 
	\item RTP-Zeitstempel: Die NTP-Zeit in Zeiteinheiten der RTP Session, die prim�r dazu dient um verschiedene Quellen zu synchronisieren.
	\item Gesendete Pakete: Die Anzahl der gesendeten Pakete sowie Bytes.
\end{itemize}

Bei RR Paket werden Werte wie Jitter, Round-Trip-Delay (Hin-und-Z�r�ck-Verz�gerung)  oder Paketverlust aus Sicht des Emfp�ngers berechnet und abgesch�tzt. 
In einem Receiver Report sind unter anderm folgende Informationen enthalten:
\begin{enumerate}
	\item Paketverlust: Der Anteil der verlorenen Pakete seit dem letzten Report, als auch die Gesamtanzahl der verlorenen Pakete.
	\item Jitter: Der Jitter wird gesch�tzt als die Varianz der Ankunftszeitdifferenzen der Pakete. 
	\item Letzter Report: Die verstrichene Zeit seid dem Empfang des letzten Reports.
\end{enumerate}

Die SDES Pakete erm�glichen es, die Quallen mit Namen zu kennzeichnen. Diese Kennzeichnung wird vom Empf�nger dazu verwendet mehrere Bitstr�me einer Quelle, die in getrennten RTP-Sessions �bertragen werden wieder zusammenzuf�hren. Das BYE Paket dient dazu das Ende der Teilnahme an einer Kommunikation anzuzeigen. 

\section{VoIP-relevante Sprachcodierungsverfahren}
Es gibt mehrere Sprachcodierungsverfahren, die bei VoIP verwendet werden k�nnen. Man unterscheided zwischen Abtastwert-orientierten Codierungsverfahren die zwar gute Sprachqualit�t garantieren aber Bitraten haben die gr��er als 16kbit/s sind und Segment-orientierten Codierungsverfahren die zwar geringe Bitraten haben aber daf�r nure einer schlecht bis guten Sprachqualit�t bieten. 
Obwohl die Qualit�t quantitativ durch Parameter wie die Laufzeit der IP-Pakete(Delay), die Laufzeitunterschiede(Jitter) und die Verlustrate von IP-Paketen im Netz beinflusst wird, sagen diese Parameter nichts �ber die eigentliche Sprachqualit�t aus. Die eigentliche Sprachqualit�t wird nach der MOS (Mean Opinion Score) - Skala  gemessen. Diese unterscheidet Aspekte wie Verst�ndlichkeit der Sprache, Akzeptanz der Lautst�rke sowie die Akzeptanz der Laufzeitschwankungen und Echos. 

%TODO TABELLE �BER MOSS S147

Eine Analoge Sprach�bertragung kommt auf einen MOS-Wert von 3.5 bis 4. Hochwertige VoIP-Implementierungen bieten eine Sprachqualit�t zwischen 3.8 und 4.4 je nach verwendeter Sprachcodierung. Dabei bieten VoIP-Systeme mindestens die Qualit�t eines herk�mmlichen analogen Telefonnetzes. 

\section{STUN} 
\section{Konferenzschaltungen}
\section{Skype}
\subsection{Registrierung und Login}
\subsection{Rufsignalisierung und Sprach�bertragung}
\subsection{Technologie von Skype}
\subsubsection{Verbindungsstruktur}
\subsubsection{Sicherheit}
\subsubsection{Konferenzen}



\chapter{Summary, Conclusions, and Further Work}
\label{chap:conclusions}
The purpose of this book is to understand  the influence of representations on the performance of genetic and evolutionary algorithms. 
This chapter summarizes the work contained in this study and lists its major contributions.

\selectlanguage{english}
\section{Summary}

We  started in Chap.~\ref{chap:einleitung} by providing the necessary background for examining representations for  GEAs. Researchers recognized early that representations have a large influence on the performance of GEAs. Consequently, after a brief introduction into representations and GEAs, we discussed how the influence of representations on problem difficulty  can be measured. The chapter ended with prior guidelines for choosing high-quality  representations. Most of them are  mainly based on empirical observations and intuition and not on theoretical analysis.

Therefore, we presented in Chap.~\ref{cha:grafiken} three aspects of a theory of representations for  GEAs. We investigated how the locality, scaling, and locality of an encoding  influences GEA performance. The performance of GEAs is determined by the solution quality at the end of a run and the number of generations until the population is converged. Consequently, for redundant and exponentially scaled encodings, we presented population sizing models and described how the time to convergence is changed.
Furthermore, we were able to demonstrate that high-locality encodings do not change the difficulty of a problem; in contrast, when using low-locality encodings, on average, the difficulty of problems changes. Therefore,  easy problems become more difficult and difficult problems become easier by the use of low-locality encodings.
For all three properties of encodings, the theoretical models were verified with empirical results.


\section{Conclusions}
We  summarize the most important contributions of this work.

{\bf Framework for design and analysis of representations (and operators) for GEAs.} The main purpose of this study was to present a  framework which describes how genetic representations influence the performance of GEAs. The performance of GEAs is measured by the solution quality at the end of the run and the number of generations until the population is converged. 
The proposed framework allows us to analyze the influence of existing representations on GEA performance and to develop efficient new representations in a theory-guided way.
Furthermore, we illustrated that the framework can also be used for the design and analysis of search operators, which are relevant for direct encodings.
Based on the framework, the development of high-quality representations remains not only a matter of intuition and random search but becomes an engineering design task.
Even though more work is needed, we believe that the results presented are sufficiently compelling to recommend increased use of the framework.



{\bf Redundancy, Scaling, and Locality}. These are the three elements of the proposed framework of representations.  We demonstrated that these three properties of representations influence GEA performance and presented theoretical models to predict how solution quality and time to convergence changes.
By examining the redundancy, scaling, and locality of an encoding, we are able to predict the influence of representations on GEA performance.

The theoretical analysis shows that the redundancy of an encoding influences the supply of building blocks (BB) in the initial population. $r$ denotes the number of genotypic BBs that represent the best phenotypic BB, and $k_r$ denotes the order of redundancy. For synonymously redundant encodings, where all genotypes that represent the same phenotype are similar to each other, the probability of GEA failure goes either with  $O(\exp(-r/2^{k_r}))$ (uniformly scaled representations) or  with $O(\exp(-\sqrt{r/2^{k_r}}))$ (exponentially scaled representations).
Therefore, GEA performance increases if the representation overrepresents high-quality BBs. If a representation is uniformly redundant, that means each phenotype is represented by the same number of genotypes, GEA performance remains unchanged in comparison to non-redundant encodings.

The analysis of the scaling of an encoding reveals that non-uniformly scaled representations modify the dynamics of genetic search. If exponentially scaled representations are used, the alleles are solved serially which increases the overall time until convergence and results in problems with genetic drift but allows rough approximations of the expected optimal solution after a few generations.

We know from previous work that the high locality of an encoding is a necessary condition for efficient mutation-based search.
An encoding has high locality if neighboring phenotypes correspond to neighboring genotypes.
Investigating the  influence of locality shows that  high-locality encodings do not change the difficulty of a problem. In contrast, low-locality encodings, where phenotypic neighbors do not correspond to genotypic neighbors, change problem difficulty and make, on average, easy problems more difficult and deceptive problems easier.
Therefore, to assure that  an easy problem remains easy, high-locality representations  are necessary.

\section{Further Work}

What are the open questions? What should be done next?

\selectlanguage{ngerman}



%%% Local Variables: 
%%% mode: latex
%%% TeX-master: "..\\da-beispiel"
%%% End: 


%\selectlanguage{ngerman} % jetzt sprechen wir wieder deutsch.

\backmatter
\bibliographystyle{literatur/natdin}
\bibliography{literatur/da}
\chapter*{Eidesstattliche Erkl\"{a}rung}\thispagestyle{empty}Ich versichere, dass ich meine Diplomarbeit ohne Hilfe Dritter und ohne Benutzunganderer als der angegebenen Quellen und Hilfsmittel angefertigt und die den benutzten Quellen w\"{o}rtlich oder inhaltlich entnommenen Stellen als solche kenntlich gemacht habe. Diese Arbeit hat in gleicher oder \"{a}hnlicher Form noch keiner Pr\"{u}fungsbeh\"{o}rde vorgelegen.\bigskip\raggedright{Mannheim, den 17.06.2008} \bigskip \bigskip \bigskip Thomas Plotkowiak

\end{document}
