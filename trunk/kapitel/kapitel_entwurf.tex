\chapter{Entwurf}
\section{Wahl der Spieleumgebung}
\subsection{Wahl der Programiersprache}

\section{Kommunikationsaspekte}
\subsection{Sprache vs. Nonverbal vs. Text}
\subssection{Proximit�t}
\subsection{Interpersonelle vs. Massenkommunikation}
\subsection{...}


\section{Wahl des VoiP Standards}
\subsection{Skype}
\subsection{SIP}
\subsubsection{PJSIP}
\subsubsection{WengoPhone}
\subsubsection{others}


\section{Wahl der Netzwerk Architektur}
\subsection{Client Server}
\subsubsection{Netmeeting}
\subsubsection{Ventrillo}
\subsubsection{Teamspeak}
\subsubsection{Battlecom}

\subsection{Reine Peer-to-Peer Modelle}
\subsection{Problems}

Client Server O(1) Lookup Latency vs. Chord O(logN) lookup!
Security Issues 


\subsection{Hybrid Model}

\subsubsection{Design alternatives}
\subsubsection[{Replicate registrations vs. search on call setup}
Durch die serverbasierte Architektur kann der server schnell ein Flaschenhals sowohl f�r die Bandbreite als auch f�r die Verf�gbarkeit werden. Eine M�glichkeit ist der einsatz mehrerer Redundanter Server. Dabei bestehen zwei Alternativen.

\begin{itemize}
	\item Die Benutzer Lokationsinformations auf allen Servern zu replizieren.
	\item Die Suche nach dem korrekten Server der die Lokation des Benutzers enth�lt auf allen Servern auszuf�hren, w�hrend die Registrierung nur auf einem Server geschieht.
\end{itemize}

% BILD see \cite{schulzrinne05}

Im ersten Fall k�nnte durch Datenbanken Replikation Sichergestellt werden dass die Benutzereintr�ge zwischen verschiedenen Servern konsistent gehalten werden. Im zweiten Fall kann entweder der Benutzer alle Server kontaktieren oder der erste kontaktiere Server die Anfrage weiterleiten. 

Der Nachteil des ersten Ansatzes ist das eine Synchronisation f�r jede Registrierung erforderlich ist. Es besteht die M�glichkeit dass Eintr�ge veraltet sind bevor eine Synchronisation zwischen den Servern ausgef�hrt wird. Mit einer wachsenden Anzahl and Benutzern k�nnte diese Architektur der zus�tzliche Netzwerkverkehr zwischen den Servern zu einem Flaschenhals werden. 

Im Anderen Fall ist die Verbindungslatenz h�her da erst eine Reihe von Schritten durchlaufen werden muss. Eine parallele Suche erh�ht zudem die ben�tigte Bandbreite.  

Beide Ans�tze k�nnen fehlschlagen wenn die Anzahl an Benutzern sehr gro� wird.

%TODO Analysis of Ansatz 1
%\cite{schulzrinne07}


