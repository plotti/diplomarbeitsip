		
\section{Minor Contributions}
Zwar ist bei der Textkommunikation der Zeitverzug weitaus h�her, schon bedingt durch den limitierenden Einfluss der pers�nlichen Schreibgeschwindigkeit der Gespr�chspartner, trotzdem werden diese Antwortzeiten nicht als Hinderniss wahrgenommen. Im Gegensatz zur Sprachkommunikation findet hier eine abstrahierte asynchrone Kommnunikation statt, w�hrend bei der Sprachkommnuikation jegliche Asynchronit�t als eine Unterbrechnung des Kommunikationsflusses gesehen wird. Bei den Kenngr��en der Interaktivit�t dominiert die Sprachkommunikation jedoch durch die direkte intuitive R�ckkanalm�glichkeit die vergleichenen Kommunikationsarten. Die Lebhaftigkeit von gesprochener Kommunikation wird weithin als hoch angesehen und so wird einzig die Video-Kommunikation und ein "`Face-to-Face"' Gespr�ch als noch lebhafter wahrgenommen. Da der Spieler mittels der Sprachkommunikation noch tiefer in das Spielgeschehen eintaucht wird das erlebte noch lebhafter wahrgenommen. Trotz ihrer technischen Einschr�nkungen vor allem in Bezug auf die interpersonelle Kommunikation in Spielen ist die Sprachkommunikation in weiten Bereichen den anderen Kommunikationsarten stark �berlegen. Da Sprache auch unsere Kommunikation im Alltag bestimmt, liegt es Nahe, dass dieses Medium auch die vorherrschende Kommunikationsmethode in Computerspielen werden kann. Dar�ber hinaus bietet sie im Vergleich zur reinen Textkommunikation noch weitere Vorteile die ihr als Alleinstellungsmerkmal dienen:
