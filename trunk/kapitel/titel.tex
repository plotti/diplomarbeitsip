% Die Titelseite der Arbeit

\begin{titlepage}

\begin{center} % zentrieren

  % Logo der Universit{\"a}t Mannheim
  \begin{figure}[ht]
    \centering
    \includegraphics{grafiken/unilogo}
  \end{figure}
  
  % Vertikaler Zwischenraum
  \bigskip
  \vfill 
  \begin{framed}
  % Titel der Arbeit und Typ der Arbeit, umrandet
    \begin{center}
      \textsc{{\Large Implementierung einer integrierten, distanzbasierten Sprachkommunikation f�r Peer-to-Peer-Spiele \\}}
                                % Letztes \\ ist wichtig, beginnt eine neue Zeile f{\"u}r die Art der Arbeit
  
      \bigskip
  
                                % Art der Arbeit, ggf. auszutauschen gegen Seminar- oder Doktorarbeit
      \textbf{Diplomarbeit}
    \end{center}
    \end{framed}
    \vfill
    \vfill
  
  % Daten des Erstellers, Einreichungsdatum
  % in einer Tabelle ausgerichtet
  \begin{tabular*}{0.62\textwidth}{r@{\extracolsep{\fill}}l}
    eingereicht : & Juni 2008\\\\
    von: & Thomas Plotkowiak\\
    & geboren am 09. Juli 1981\\
    & in Zabrze\\
    \\
    Matrikelnummer: & 0933679\\
    \\
    Betreuer: & Tonio Triebel \\
    & Dr. rer. nat. Stephan Kopf\\
  \end{tabular*}
  \vfill
  \vfill
  
  % Unten: Kontaktdaten des Lehrstuhls f{\"u}r Wirtschaftsinformatik 1
  
  \rule{\textwidth}{.4pt}\\ % vertikale Linie
  Universit{\"a}t Mannheim\\
  Lehrstuhl f{\"u}r Praktische Informatik IV\\
  D -- 68131 Mannheim\\
  Telefon: +49 621 181 2600, Fax +49 621 181 2601\\
  Internet: \url{http://www.informatik.uni-mannheim.de/pi4/}
\end{center}

\end{titlepage} % Ende des Titelblatts

%%% Local Variables: 
%%% mode: latex
%%% TeX-master: "~/Documents/DA-Vorlage/beispiel/da-beispiel"
%%% End: 
