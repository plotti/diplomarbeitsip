\chapter{Einleitung}
\label{chap:einleitung}

%TODO REWRITE.

Computerspiele entwickelten sich in circa 50 Jahren von eher technischen Versuchen an Universit�ten zu einer der einflussreichsten Freizeitgestaltungsformen des 21. Jahrhunderts und sind heute zu einem Medium geworden, in dem sich tausende von Menschen aus der ganzen Welt treffen und miteinander spielen. Der Umsatz der weltweiten Computerspielbranche kletterte 2003 nach Angaben mehrerer Marktforschungsunternehmen wie JISC auf den Rekordwert von 19,2 Mrd. Euro und erreichte damit fast das Niveau der Filmindustrie. %\cite{printout1}

Eine Kommunikation in solchen Spielen war zun�chst nicht vorhanden, doch mit dem Aufkommen des Mehrspieler-Spiels entstand die erste Stufe der Kommunikation(der Chat). Datenkommunikation Beim "Chatten" werden lediglich Textnachrichten zwischen einem oder mehreren Mitspielern ausgetauscht. Oft wurden Mehrspieler-Spiele auch nur innerhalb eines LANs gespielt, so dass immer noch eine direkte Sprachkommunikation zwischen den Mitspielern m�glich war, weil sie sich z.B. immer noch im gleichen Raum oder Geb�ude befanden. 

Mit dem Aufkommen von Mehrspieler-Spielen wurde es auch zum ersten Mal n�tig die Spielezust�nde der Mitspieler miteinander auszutauschen. Dies f�hrte zur Einf�hrung der Client- und Server-Architektur, in der ein zentraler Server die Spielelogik verwaltete und an Clients verteilte. Da man sich in LANs befand, musste ein solcher Server die Daten von 2 bis zu 1000 Spielern verwalten. 

Doch mit dem Aufkommen der MMORPGs (Massive(ly) Multiplayer Online Role-Playing Games), bei denen Mitspieler nun durch das Internet miteinander spielen und kommunizieren, kam es zu neuen Herausforderungen: 
Einerseits wurde eine lokale Sprachkommunikation unm�glich, da sich Spieler des gleichen Spiels auf verschiedenen Kontinenten befinden und andererseits mussten die zentralen Spieleserver nun Daten von bis zu 10 Millionen Spielern verwalten und verteilen. 

%Mundane metaverses communicating vy voice in virtual worlds ---> INTRO
%\cite{wadley08}

\section{Ziel der Arbeit}

In dieser Diplomarbeit sollen ein P2P basiertes Verfahren zur Sprachkommunikation f�r Spiele entwickelt werden. Statt propri�teren L�sungen wie Skype, soll ein offenes und bereits etabliertes Protokoll genutzt werden. Dabei sollen und Probleme und Chancen des Einsatzes eines solchen Protokolls zu er�rtert und L�sungsvorschl�ge erarbeitet werden. Zum Einsatz soll das SIP Protokoll kommen, das bisher im Bereich des VoIP zu einem offenen Standard entwickelt hat, der auf Software- und Hardware-Ebene seit mehreren Jahren genutzt wird. Obwohl es sich im Ansatz um ein Client-Server-rotokoll handelt, bietet SIP zahlreiche M�glichkeiten es auch im P2P Modus zu betreiben, und es finden sich viele An�tze das Protokoll komplett dezentral einzusetzen.  
Bisher ist der Einsatz von SIP im Bereich von Multispieler-Spielen noch nicht tiefgehend erforscht. Es soll  untersucht werden wie das SIP-Protokoll sinnvoll f�r eine standardisierte P2P-Schnittstelle zwischen Clients eingesetzt werden kann. Ebenfalls soll der zuk�nftige Einsatz von P2PSIP soll untersucht werden...TODO.
Es soll eine Spiele-Implementierung entwickelt werden, die mit dem SIP Protokoll, den Datenaustausch zwischen Spielern erm�glicht. Dieser soll mit Hilfe des SIP-SIMPLE-Protokolls implementiert werden.Die Mehrspieler-Kommunikation, die bisher mit propri�terer Software gel�st wird und nicht mit der Spielelogik verzahnt ist, soll im Rahmen der Diplomarbeit in das Spiel integriert werden. So sollen zwischen Mitspielern dynamisch Konferenzen aufgebaut werden, falls sich diese in einer zu definierten "H�rn�he" befinden. Falls sich ein Mitspieler au�erhalb der H�rn�he befindet, sollen entsprechende Konferenzen abgebaut werden, wobei zu �berlegen ist, aber verschiedene Verbindungsstufen aufrechtzuerhalten. Zur Abblidung von Lokalen Audioereignissen soll eine weiteren Audiorendering-Engine in das Spiel integriert werden, die es erm�glichen kann  lokale Soundquellen zu positionieren und Audiodaten, die nicht �ber das Netzwerk zu �bertragen sind, lokal zu verwalten. Die Implementierung soll mit Skype und Multicast-Sockets Netzwerklayern verglichen werden, es soll untersucht werden inwiefern die verschiedenen zugrunde liegenden Systeme miteinander vergleichbar sind. Unter diesem Aspekt soll analysiert werden, wie sich die Performace dieser Systeme in einem direkten Benchmark miteinander verh�lt.


\section{Aufbau der Arbeit}
\label{sec:aufbau-der-arbeit}

Hier den Aufbau der Arbeit erl{\"a}utern.

%%% Local Variables: 
%%% mode: latex
%%% TeX-master: "..\\da-beispiel"
%%% End: 
