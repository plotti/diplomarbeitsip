\thispagestyle{empty}
\vspace*{\fill}
\section*{\centering \abstractname}
\begin{raggedsection}
\begin{centering}
\small
Bisherige Sprachkommunikationsl�sungen f�r Mehrspieler-Computerspiele sind nicht in diese integriert, serverbasiert und verf�gen oft �ber nur einen Kommunikationskanal. Dadurch sind solche L�sungen schwer bedienbar, bieten wenig Kontrolle �ber die Konversation und k�nnen zudem f�r Betreiber der Server hohe Kosten verursachen. In dieser Arbeit wird eine Implementierung einer integrierten, distanzbasierten Sprachkommunikation f�r Spiele entwickelt, die im Gegensatz zu bisherigen L�sungen, keinen zentralen Konferenz-Server zum Mischen des Audiostroms benutzt. Ein solcher Ansatz hilft Kosten zu sparen, indem Audioverbindungen auf Peer-to-Peer-Basis erfolgen und so die Bandbreite und Rechenleistung von Teilnehmern gestellt wird. Dadurch erh�lt der Spieleclient auch die lokale Kontrolle �ber alle eingehenden Audiostr�me und kann durch einen distanzbasierten Audio-Mischvorgang f�r den Spieler die Metapher der Luft�bertragung von Sprache erzeugen, die ihm eine intuitive Sprachkommunikation erm�glicht. Diese Metapher bietet die Ausgangsbasis, um Konzepte der Proxemik aus der wirklichen Welt analog im virtuellen Raum umzusetzen und dort einen maximalen H�rradius zu definieren. Dadurch werden nicht mehr mit jedem Teilnehmer Verbindungen aufgebaut und Bandbreite kann so eingespart werden. Die beschriebenen Konzepte werden in einem 3D-Echtzeit-Spiel-Prototypen umgesetzt, indem eine 3D-Engine mit einem auf dem SIP-Protokoll basierenden VoIP-Protokollstapel kombiniert wird. �ber den Fokus der Arbeit hinausgehend wird auch aufgezeigt, wie das SIP-Protokoll als Netzwerkschnittstelle f�r Mehrspieler-Computerspielen eingesetzt werden kann. \end{centering}
\end{raggedsection}
\vfill