\thispagestyle{empty}
\vspace*{\fill}
\section*{\centering \abstractname}
\begin{raggedsection}
\begin{centering}
\small
Die vorliegende Arbeit entwickelt einen verbesserten, integrierten, distanzbasierten Ansatz f�r Sprachkommunikation in Mehrspieler-Computerspielen basierend auf einer Evaluation bestehender L�sungen. Bisher bieten diese wenig Kontrolle �ber die Konversation und sind mit hohen Betreiberkosten verbunden, da sie nicht in das Spiel integriert und serverbasiert sind. Im Gegensatz dazu werden in dieser Arbeit Audioverbindungen auf Peer-to-Peer-Basis eingesetzt, die Bandbreite und Rechenleistung der Teilnehmer nutzen und ihnen daf�r die lokale Kontrolle �ber alle eingehenden Audiostr�me erlauben. Durch einen distanzbasierten Audio-Mischvorgang wird die Metapher der Luft�bertragung von Sprache im Spiel erzeugt, die f�r den Spieler eine intuitive Sprachkommunikation erm�glicht. Basierend auf dieser Metapher werden Konzepte der Proxemik aus der wirklichen Welt analog im virtuellen Raum umgesetzt und dort ein maximaler H�rradius definiert. Der daraus resultierende Aufbau von relevanten Audioverbindungen und die Anpassung der Bitrate des Audiosignals an die Entfernung zu anderen Teilnehmern f�hren so zu effektiven Bandbreiteneinsparungen. Durch die Kombination einer 3D-Engine mit einem auf dem SIP-Protokoll basierenden VoIP-Protokollstapel wird ein 3D-Echtzeitspiel-Prototyp erstellt, der die Umsetzbarkeit dieser Konzepte demonstriert.
 \end{centering}
\end{raggedsection}
\vfill