\chapter{Verwandte Arbeiten}

\section{Work of Farzad Safei and Paul Boustead}
	\cite{safei04}
	\cite{safei04b}
	\cite{safei06b}
	\cite{safei05}
	\cite{safei06}
	\cite{safei07}

\section{Work of Kevin Hew, Martin Gibbs, Greg Wadley}
	\cite{gibbs06}
	\cite{Gibbs04a}
	\cite{gibbs04}
	
\section{Work of Aameek Singh -- SIP for Games}
	\cite{singh04}
	\cite{singh05}
	\cite{singh06}
	
\section{Work of Schulzrinne -- Full mesh conferencing}
	\cite{schulzrinne02}
	\cite{schulzrinne03}
	\cite{schulzrinne05}
			
\section{Minor Contributions}
\subsection{Work of Tonio Triebel -- Skype for Games }
	\cite{triebel07}
\subsection{Work of Xiaohui Gu }
	\cite{gu05}
\subsection{Work of Bolot }
	\cite{bolot98}
\subsection{Work of Radenkovic}
	\cite{radenkovic02}
\subsection{Work of Stadler and Miladinovic}
	%unpublished
	\cite{miladinovic-multiparty} 


Da die Sprachkommunikation in Echtzeit verl�uft, gilt die Wiedergabe eines Sprachsignals am Ziel asl qualitativ schlecht, wenn sie einem zu gro�en Zeitverzug erfolgt. F�r die Ende-zu-Ende-Verz�gerung $T_{EE}$(End-to-End-Delay) des Sprachsignals werden daher Grenzwerte gesetzt. Nach dem ITU-T-Dokument G.114 wird die VoIP-Qualit�t wie flogt klassifiziert:
\begin{itemize}
	\item $T_{EE}$ kleiner als 150ms: akzeptabel f�r alle Benutzer,
	\item $T_{EE}$ zwischen 150ms und 300ms: akzeptabel, aber mit Einschr�nkungen (nicht f�r empfindliche Benutzer), 
	\item $T_{EE}$ gr��er als 300ms: nicht akzeptabel. 
\end{itemize}

Zwar ist bei der Textkommunikation der Zeitverzug weitaus h�her, schon bedingt durch den limitierenden Einfluss der pers�nlichen Schreibgeschwindigkeit der Gespr�chspartner, trotzdem werden diese Antwortzeiten nicht als Hinderniss wahrgenommen. Im Gegensatz zur Sprachkommunikation findet hier eine abstrahierte asynchrone Kommnunikation statt, w�hrend bei der Sprachkommnuikation jegliche Asynchronit�t als eine Unterbrechnung des Kommunikationsflusses gesehen wird. Bei den Kenngr��en der Interaktivit�t dominiert die Sprachkommunikation jedoch durch die direkte intuitive R�ckkanalm�glichkeit die vergleichenen Kommunikationsarten. Die Lebhaftigkeit von gesprochener Kommunikation wird weithin als hoch angesehen und so wird einzig die Video-Kommunikation und ein "`Face-to-Face"' Gespr�ch als noch lebhafter wahrgenommen. Da der Spieler mittels der Sprachkommunikation noch tiefer in das Spielgeschehen eintaucht wird das erlebte noch lebhafter wahrgenommen. Trotz ihrer technischen Einschr�nkungen vor allem in Bezug auf die interpersonelle Kommunikation in Spielen ist die Sprachkommunikation in weiten Bereichen den anderen Kommunikationsarten stark �berlegen. Da Sprache auch unsere Kommunikation im Alltag bestimmt, liegt es Nahe, dass dieses Medium auch die vorherrschende Kommunikationsmethode in Computerspielen werden kann. Dar�ber hinaus bietet sie im Vergleich zur reinen Textkommunikation noch weitere Vorteile die ihr als Alleinstellungsmerkmal dienen: